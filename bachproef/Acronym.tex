%Packages
\usepackage[utf8]{inputenc}
\usepackage[acronym]{glossaries}

\makeglossaries

%Áfkoringen

\newacronym{dect}{DECT}{Digital Enhanced Cordless Telecommunications}
\newacronym{poc}{poc}{Proof of concept}
\newacronym{lte}{LTE}{Long Term Evolution}
\newacronym{qos}{QOS}{Quality of Service}
\newacronym{ran}{RAN}{Radio Access Network}
\newacronym{o-ran}{O-RAN}{Open Radio Access Network}
\newacronym{5gnr}{5G NR}{Fifth Generation New Radio}
\newacronym{oam}{OAM}{Operations, Administration \& Maintenance}
\newacronym{osi}{OSI}{Open Systems Interconnection}
\newacronym{ule}{ULE}{Ultra low energy}
\newacronym{cat-iq}{CAT-iq}{Cordless Advanced Technology \-internet \& quality}
\newacronym{pstn}{PSTN}{Public Switched Telephone Network}
\newacronym{voipsa}{VOIPSA}{Voice over IP Security Alliance}
\newacronym{cia}{CIA}{confidentiality, integrity, availability}
\newacronym{b2h}{B2H}{Blockchain to Healthcare}
\newacronym{bits}{BITS}{Blockchain-driven Intelligent Scheme for Telesurgery System}
\newacronym{gdpr}{GDPR}{General Data Protection Regulation}
\newacronym{nis2}{NIS2}{Network and Information Security 2-richtlijn}
\newacronym{ehds}{EHDS}{European Healthcare Data Space}
\newacronym{sdr}{SDR}{Software Defined Radio}
\newacronym{ueransim}{UERANSIM}{open source state-of-the-art 5G UE and RAN (gNodeB) simulator}
\newacronym{ue}{UE}{User Equipment}
\newacronym{srsran}{srsRAN}{RAN with SRS}
\newacronym{srs}{SRS}{Software Radio Systems}
\newacronym{e2e}{E2E}{End to end}
\newacronym{pdu}{PDU}{Protocol Data Unit}
\newacronym{udr}{UDR}{Unified Data Repository}
\newacronym{sa}{SA}{Stand-alone}
\newacronym{nsa}{NSA}{Non Stand-alone}
\newacronym{nas}{NAS}{Non-Access Stratum}
\newacronym{ngap}{NGAP}{Next Generation Application Protocol}
\newacronym{sbi}{SBI}{Service Based Interface}
\newacronym{mvp}{MVP}{Minimu Viable Product}


%Woordverklaring

\newglossaryentry{srsRAN}
{
    name=srsRAN,
    description={srsRAN is een software suite. Het is een open-source collectie van 4G en 5G software radio implementaties van SRS. \textcite{McAuliffe2023}},
},

\newglossaryentry{Open5GS}
{
    name=Open5GS,
    description={Open-source software that simulates the 5G core network. \textcite{Open5GS2023}},
},

\newglossaryentry{label}
{
    name=Name,
    description={Description},
},
%Print
\chapter{\IfLanguageName{dutch}{Afkortingen}{Acronyms}}%
\label{ch:Acronym}

\printglossary[type=\acronymtype]

\chapter{\IfLanguageName{dutch}{Woordverklaring}{Glossary}}%
\label{ch:Glossary}

\printglossary