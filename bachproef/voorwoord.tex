%%=============================================================================
%% Voorwoord
%%=============================================================================

\chapter*{\IfLanguageName{dutch}{Woord vooraf}{Preface}}%
\label{ch:voorwoord}

% Alarmen mag je zeker niet negeren, zeker niet als je moe bent. 

% Door mijn studies heen heb ik altijd een praktische kijk gehad op andere sectoren. Vaak met het idee, kunnen we dit niet verbeteren met nieuwe technologieën. Dus toen ik de kans kreeg om een bachelorproef te schrijven over een onderwerp dat nieuwe technologie te gebruiken om een maatschappelijke impact te hebben, was ik meteen enthousiast.

% Het doel van deze bachelorproef is om het zorgverlenend personeel te ondersteunen met een nieuw communicatie systeem dat de stress bij zorgverleners vermindert en hierdoor hun focus en aandacht verbeterd, met een betere patiëntenzorg en efficiëntere ziekenhuiswerking tot gevolg.

% Dit onderzoek biedt me de kans om mijn kennis mijn theoretische kennis in netwerken te verdiepen in het mobiele netwerk aspect. Het is de synergie tussen de zorg en de technologie die mij aanspreekt. Het idee dat de bachelorproef een directe impact kan hebben op de zorgopleiding en zorgsector is een extra motivatie. 

% Deze bachelorproef maakt deel uit van het 5Genius project. Dit is een samenwerkingsproject tussen HOGENT en CityMesh, met steun van Europese Unie. Beide partijen hebben mij ondersteund op hun eigen manier. HOGENT heeft het 360° Zorglab naar voor geschoven als niet-technische ondersteuning. 360° Zorglab was voor mij als IT-student een raam, langswaar ik naar de werking en het gebeuren binnen een (gesimuleerde) zorgomgeving kon gadeslaan. CityMesh heeft mij technische ondersteuning geboden als partner in het 5Genius project. Zij hebben, in hun samenwerking met HOGENT, mij hardware en de kennis, die nodig was om dit project tot een goed einde te brengen, aangerijkt.

% Ik wens specifiek mij co-promotoren, Dhr. T. Cuelenaere, Dhr. J. Buysse en Dhr. M. De Dekoninck, te bedanken voor hun begeleiding en ondersteuning. Hun kritische feedback en constructieve bijdragen hebben een belangrijke rol gespeeld in de totstandkoming van dit onderzoek. Mijn promotor Mevr. L. De Mol, voor haar begeleiding, geduld en overzicht dat zij geboden heeft die ik soms miste.
% Mijn ouders en zus als spellingscontrole en testpubliek. Ik denk dat zij intussen ook het 5G systeem goed kennen.

Alarmen mag je zeker niet negeren, zeker niet als je moe bent. Hoe kan je nu IT inzetten om stress bij zorgverleners te verminderen? Dat is wat deze bachelorproef probeert te beantwoorden.
\\
Door mijn studies heen heb ik altijd een praktische kijk gehad op andere sectoren. Vaak met het idee, kunnen we dit niet verbeteren met nieuwe technologieën. Dus toen ik de kans kreeg om een bachelorproef te schrijven over een onderwerp dat nieuwe technologie gebruikt om een maatschappelijke impact te hebben, was ik meteen enthousiast.
\\\\
Het doel van deze bachelorproef is om het zorgverlenend personeel te ondersteunen met een nieuw communicatiesysteem dat de stress bij zorgverleners vermindert en hierdoor hun focus en aandacht verbetert, met een betere patiëntenzorg en efficiëntere ziekenhuiswerking tot gevolg.
\\
Dit onderzoek biedt me de kans om mijn theoretische kennis van netwerken te verdiepen in het mobiele netwerk aspect. Het is de synergie tussen de zorg en de technologie die mij aanspreekt. Het idee dat de bachelorproef een directe impact kan hebben op de zorgopleiding en zorgsector is een extra motivatie.
\\\\
Deze bachelorproef maakt deel uit van het 5Genius project. Dit is een samenwerkingsproject tussen HOGENT en CityMesh, met steun van de Europese Unie. Beide partijen hebben mij ondersteund op hun eigen manier. HOGENT heeft het 360° Zorglab naar voren geschoven als niet-technische ondersteuning. 360° Zorglab was voor mij als IT-student een raam, waarlangs ik naar de werking en het gebeuren binnen een (gesimuleerde) zorgomgeving kon gadeslaan. CityMesh heeft mij technische ondersteuning geboden als partner in het 5Genius project. Zij hebben, in hun samenwerking met HOGENT, mij hardware en de kennis, die nodig was om dit project tot een goed einde te brengen, aangereikt.
\\\\
Ik wens specifiek mijn co-promotoren, Dhr. T. Cuelenaere, Dhr. J. Buysse en Dhr. M. De Dekoninck, te bedanken voor hun begeleiding en ondersteuning. Hun kritische feedback en constructieve bijdragen hebben een belangrijke rol gespeeld in de totstandkoming van dit onderzoek. Mijn promotor Mevr. L. De Mol, voor haar begeleiding, geduld en overzicht dat zij geboden heeft die ik soms miste.
Mijn ouders en zus als spellingscontrole en testpubliek. Ik denk dat zij intussen ook het 5G-systeem goed kennen.

%% TODO:
%% Het voorwoord is het enige deel van de bachelorproef waar je vanuit je
%% eigen standpunt (``ik-vorm'') mag schrijven. Je kan hier bv. motiveren
%% waarom jij het onderwerp wil bespreken.
%% Vergeet ook niet te bedanken wie je geholpen/gesteund/... heeft
