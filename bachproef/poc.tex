\chapter{\IfLanguageName{dutch}{POC: Simulatie}{POC: Simulation}}%
\label{ch:poc}

\section{\IfLanguageName{dutch}{Simulation}{Simulation}}%
\label{sec:sim}%
De \gls{poc} is een simulatie van een 5G-netwerk. Dit wordt opgezet door gebruik te maken van \gls{open5gs}. De \gls{vm} met als \gls{os} Ubuntu, versie 22.04.5 LTS ook wel gekend als Jammy. 

De opbouw van de simulatie zal in 2 fases gebeuren. De eerste is een volledig manuele build. Hier wordt een 5G-netwerk gesimuleerd op een laptop. Dit netwerk bestaat uit een \gls{gnb} en een \gls{ue}. Beiden zullen op eenzelfde toestel in een \gls{vm} worden gerund. De 2de fase is het omzetten van deze manuele build naar een automatische deployment van de \gls{poc}

Als basis hardward configuratie is er gekozen voor een \gls{vm} met 8 Gigabyte RAM en 2 CPU cores. Verder is de netwerkkaart geconfigureerd op het NAT netwerk van VirtualBox zelf.

\section{\IfLanguageName{dutch}{Installatie}{Installation}}%
\label{sec:installation}%

De installatie bestaat uit 3 delen:

\begin{itemize}
    \item Vereiste software
    \item Open5GS
    \item UERANSIM
\end{itemize}

De basis installatie van Open5GS gebeurt volgens de Quickstart van \gls{open5gs}, \textcite{Lee2025a}:

\begin{enumerate}
    \item Installatie van afhankelijkheden/vereisten
    \item Installatie van \gls{open5gs}
    \item Configuratie van \gls{open5gs}
    \item Starten van de \gls{open5gs} services
\end{enumerate}

Nadien wordt \gls{ueransim} geinstalleerd om de simulatie te finaliseren. 

De volledige installatie gids is ook toegevoegd in de bijlagen, zie bijlage \ref{ch:InstallationGuide}.

Er zijn ook eigen configuratie-aanpassingen gebeurt om deze \gls{poc} automatisch te deployen via Vagrant. Hiervoor is er gebruikgemaakt van een Vagrantfile. Deze bevat de hardware configuratie en een link naar het installatiescript voor de simulatie. Echter some configuratie aanpassingen moeten nog steeds manueel worden uitgevoerd.

\begin{lstlisting}[language=sh, caption=Vagrantfile]
Vagrant.configure("2") do |config|
    config.vm.provider "virtualbox" do |vb|
    # Stel de gedeelde map in om automatisch te mounten
    vb.customize ["sharedfolder", "add", :id, "--name", "vagrant_data", "--hostpath", ".", "--automount"]
    end

    config.vm.define "conductor" do |cond|
        # This is the base image for the VM - do not change this!
        cond.vm.box = "gusztavvargadr/ubuntu-server"
        cond.vm.box_version = "2404.0.2503"
        # Set the host name of the VM
        cond.vm.hostname = "Centrale"
        # Set portforwarding
        cond.vm.network "forwarded_port", guest: 9999, host: 9999, host_ip: "127.0.0.1", id: "open5gs"
        # VirtualBox specific configuration
        cond.vm.provider "virtualbox" do |vb|
            # VirtualBox Display Name
            vb.name = "Centrale"
            # VirtualBox Group
            vb.customize ["modifyvm", :id, "--groups", "/BPoc"]
            # 1GB vRAM
            vb.memory = "8192"
            # 1vCPU
            vb.cpus = "2"
        end
        cond.vm.provision "shell", path: "script-cent.sh"
    end
end
\end{lstlisting}

Om deze \gls{vm} te maken wordt er gebruikgemaakt van het commando: \verb|vagrant up|'. Echter door het meermaals testen van de \gls{vm} is er gekozen om een script te schrijven in powershell. Dit script verwijdert de vorige \gls{vm} en start de logging van de terminal. Dit is handig om te detecteren of de \gls{vm} klaar is met de configuratie, maar ook om mogelijk fouten sneller te detecteren.

\begin{lstlisting}[language=sh, caption=Build Script]
Start-Transcript -Path "vagrant.log"
vagrant destroy -f
vagrant up
Stop-Transcript
\end{lstlisting}



\section{\IfLanguageName{dutch}{Configuratie}{Configuration}}%
\label{sec:Config}%

De automatische configuratie wordt gedeployed door het oproepen van een script. Dit is geschreven in bash voor linux, terug te vinden in de bijlagen (hoofdstuk \ref{sec:script}).

Het einde van de automatische configuratie kan worden gedetecteerd aan de hand van de log file ide woerdt gegenereerd tijdens de installatie in de githubrepo. Deze log fie, vagrant.log, is beschikbaar door de shared folder die wordt aangemaakt in het begin van de \gls{vm} installatie. 

Eenmaal de configuratie klaar is moet enkel de handmatige confiuratie gebeuren.

De manuale configuratie bestaat uit X delen:

\begin{itemize}
    \item IP check
    \item Open5GS amf configuratie
    \item UE configuratie
    \item gNB configuratie
\end{itemize}

Eerst wordt er gekeken naar de huidige IP van de \gls{vm}, dit via het \verb|ip a| commando. Hier kan men het IP adres van de netwerkkaar eth0. Dit IP adres zal worden gebruikt voor verdere configuratie. In het geval van de\gls{poc}, is dit 10.0.2.15

Eenmaal het IP adres gekend is worden de hierboven opgelijste configuraties aangepast, beginnend met de amf. Hierbij wordt op lijn 23 het ip adres 127.0.0.5 naar het gevonden ip adres 10.0.2.15. Dit is het IP  adres van de ngap server.




% Installation mongodb

% missing part: 

% `echo "deb [ signed-by=/usr/share/keyrings/mongodb-server-8.0.gpg ] http://repo.mongodb.org/apt/debian bookworm/mongodb-org/8.0 main" | sudo tee /etc/apt/sources.list.d/mongodb-org-8.0.list`
% after curl before updte
