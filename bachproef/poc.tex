\chapter{\IfLanguageName{dutch}{POC: Simulatie}{POC: Simulation}}%
\label{ch:poc1}

%TODO:
De \gls{poc} is een simulatie van een 5G-netwerk. Dit wordt opgezet door gebruik te maken van \gls{open5gs}. De \gls{vm} met als \gls{os} Ubuntu, versie 22.04.5 LTS ook wel gekend als Jammy. 

De opbouw van de simulatie zal in 2 fases gebeuren. De eerste is een volledig manuele build. Hier wordt een 5G-netwerk gesimuleerd op een laptop. Dit netwerk bestaat uit een \gls{gnb} en een \gls{ue}. Beiden zullen op eenzelfde toestel in een \gls{vm} worden gerund. De 2de fase is het omzetten van deze manuele build naar een automatische deployment van de \gls{poc}

Als basis hardward configuratie is er gekozen voor een \gls{vm} met 8 Gigabyte RAM en 2 CPU cores. Verder is de netwerkkaart geconfigureerd op het NAT netwerk van 


De basis installatie gebeurt volgens de Quickstart van \gls{open5gs} \textcite{Lee2025a}:

\begin{enumerate}
    \item Installatie van afhankelijkheden/vereisten
    \item Installatie van \gls{open5gs}
    \item Configuratie van \gls{open5gs}
    \item Starten van de \gls{open5gs} services
    \item Configuratie van de \gls{ue}
\end{enumerate}

Er zijn ook eigen configuratie-aanpassingen gebeurt om deze \gls{poc} automatisch te deployen via Vagrant


Installation mongodb

missing part: 

`echo "deb [ signed-by=/usr/share/keyrings/mongodb-server-8.0.gpg ] http://repo.mongodb.org/apt/debian bookworm/mongodb-org/8.0 main" | sudo tee /etc/apt/sources.list.d/mongodb-org-8.0.list`
after curl before updte

congif vm: 8gb 2 cpu debian 12 bookwormn