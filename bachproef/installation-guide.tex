\begin{center}
    \large\textbf{Code Disclaimer}
\end{center}

\vspace{0.5cm}

\noindent De code in dit hoofdstuk is grotendeels afkomsting van de Quickstart guide van \gls{open5gs} \autocite{Lee2025a}. De code is aangepast om te voldoen aan de noden van deze bachelorproef. Indien de configuratie is aangepast of er eigen code wordt gebruikt zal dit expliciet vermeld worden.
\vspace{0.3cm}

\section{\IfLanguageName{dutch}{Installatie van afhankelijkheden/vereisten}{Installation of dependencies/requirements}}

Om \gls{open5gs} te installeren, zijn er een aantal afhankelijkheden nodig. Volgende software pakketten zijn vereist:

\begin{itemize}
    \item Databank
\end{itemize}

Op deze \gls{vm} wordt er een MongoDB databank geinstallleerd met als functie: \gls{udr}.

% \begin{listing}
%     \listingcode{bash}{Installatie van MongoDB}
%     sudo apt update
%     sudo apt install gnupg
%     curl -fsSL https://pgp.mongodb.com/server-6.0.asc | sudo gpg -o /usr/share/keyrings/mongodb-server-6.0.gpg --dearmor

%     echo "deb [ arch=amd64,arm64 signed-by=/usr/share/keyrings/mongodb-server-6.0.gpg] https://repo.mongodb.org/apt/ubuntu jammy/mongodb-org/6.0 multiverse" | sudo tee /etc/apt/sources.list.d/mongodb-org-6.0.list

%     sudo apt update
%     sudo apt install -y mongodb-org

%     sudo systemctl start mongod 
%     sudo systemctl enable mongod
% \end{listing}

\begin{lstlisting}[language=Bash, caption=Installatie van MongoDB]
    sudo apt update
    sudo apt install gnupg
    curl -fsSL https://pgp.mongodb.com/server-6.0.asc | sudo gpg 
        -o /usr/share/keyrings/mongodb-server-6.0.gpg --dearmor

    echo "deb [arch=amd64,arm64 signed-by=/usr/share/keyrings/mongodb
        -server-6.0.gpg] https://repo.mongodb.org/apt/
        ubuntu jammy/mongodb-org/6.0 multiverse" | sudo tee /etc/apt
        /sources.list.d/mongodb-org-6.0.list

    sudo apt update
    sudo apt install -y mongodb-org

    sudo systemctl start mongod 
    sudo systemctl enable mongod
\end{lstlisting}

Om te controlleren of de MongoDB service draait, kan er gebruik gemaakt worden van volgend commando: \textit{Eigen code}

% \begin{listing}
%     \listingcode{bash}{Controle MongoDB status}
%     sudo systemctl status mongod

%     # Verwacht resultaat:
%     ● mongod.service - MongoDB Database Server
%     Loaded: loaded (/lib/systemd/system/mongod.service; enabled; vendor pr>
%     Active: active (running) since Fri 2025-02-21 09:55:54 UTC; 24s ago
%     Docs: https://docs.mongodb.org/manual
%     Main PID: 2197 (mongod)
%     Memory: 68.3M
%     CPU: 159ms
%     CGroup: /system.slice/mongod.service
%             └─2197 /usr/bin/mongod --config /etc/mongod.conf
% \end{listing}

\begin{lstlisting}[language=Bash, caption=Controle MongoDB status]
    sudo systemctl status mongod

    #Verwacht resultaat:
    mongod.service - MongoDB Database Server
    Loaded: loaded (/lib/systemd/system/mongod.service; enabled; vendor pr>
    Active: active (running) since Fri 2025-02-21 09:55:54 UTC; 24s ago
    Docs: https://docs.mongodb.org/manual
    Main PID: 2197 (mongod)
    Memory: 68.3M
    CPU: 159ms
    CGroup: /system.slice/mongod.service
            |_2197 /usr/bin/mongod --config /etc/mongod.conf
\end{lstlisting}

\section{\IfLanguageName{dutch}{Installatie van Open5GS}{Installation of Open5GS}}

De \gls{open5gs} pakketten voor Ubuntu zijn beschikbaar op OBS. 

\textit{\textbf{Opgepast: Het is belangrijk dat MongoDB al geinstalleerd voor de \gls{open5gs} installatie wordt gestart.}} \autocite{Lee2025a}

% \begin{listing}
%     \listingcode{bash}{Installatie van Open5GS}
%     sudo apt update

%     wget -qO - https://download.opensuse.org/repositories/home:/acetcom:/open5gs:/latest/Debian_10/Release.key | sudo apt-key add -
%     sudo sh -c "echo 'deb http://download.opensuse.org/repositories/home:/acetcom:/open5gs:/latest/Debian_10/ ./' > /etc/apt/sources.list.d/open5gs.list"
    
%     sudo apt update
%     sudo apt install open5gs    
% \end{listing}

\begin{lstlisting}[language=Bash, caption=Installatie van Open5GS]
    sudo apt update

    wget -qO - https://download.opensuse.org/repositories/home:/acetcom:
        /open5gs:/latest/Debian_10/Release.key | sudo apt-key add -
    sudo sh -c "echo 'deb http://download.opensuse.org/repositories/home:
        /acetcom:/open5gs:/latest/Debian_10/ ./' 
        > /etc/apt/sources.list.d/open5gs.list"
    
    sudo apt update
    sudo apt install open5gs
\end{lstlisting}

\section{\IfLanguageName{dutch}{Installatie van de web-interface van \gls{open5gs}}{Installation of the web-interface of \gls{open5gs}}}

Voor de installatie van \gls{open5gs} wordt volgende code uitegevoerd:

\begin{lstlisting}[language=Bash, caption=Installatie van web-interface]
    # Download and import the Nodesource GPG key
    sudo apt update
    sudo apt install -y ca-certificates curl gnupg
    sudo mkdir -p /etc/apt/keyrings
    curl -fsSL https://deb.nodesource.com/gpgkey/nodesource-repo.gpg.key | sudo gpg --dearmor -o /etc/apt/keyrings/nodesource.gpg
    
    # Create deb repository
    NODE_MAJOR=20
    echo "deb [signed-by=/etc/apt/keyrings/nodesource.gpg] https://deb.nodesource.com/node_$NODE_MAJOR.x nodistro main" | sudo tee /etc/apt/sources.list.d/nodesource.list
    
    # Run Update and Install
    sudo apt update
    sudo apt install nodejs -y

    # Install the web interface
    curl -fsSL https://open5gs.org/open5gs/assets/webui/install | sudo -E bash -
\end{lstlisting}

\section{\IfLanguageName{dutch}{Configuratie van \gls{open5gs}}{Configuration of \gls{open5gs}}}



\subsection{\IfLanguageName{dutch}{Configuratie van een 5G Core}{Configuration of a 5G Core}}



\subsection{\IfLanguageName{dutch}{Configuratie van logging systeem}{Configuration of logging system}}



\section{\IfLanguageName{dutch}{Activatie van eNB/gNB en UE}{Activation of eNB/gNB and UE}}

\section{\IfLanguageName{dutch}{Bash script voor vagrant deployment}{Bash script for vagrant deployment}}
\label{sec:script}

Dit is het script dat wordt gebruikt door vagrant om de simulatie te installeren en configureren op de \gls{vm}:

\begin{lstlisting}[language=Bash, caption=Script]
    # Script to install Open5GS and UERANSIM on Ubuntu 22.04
    #Device: Conductur / Centrale
    #-----------------------------------------------------------
    #Install MongoDB for Open5GS
    sudo apt-get install gnupg curl
    curl -fsSL https://www.mongodb.org/static/pgp/server-8.0.asc | sudo gpg -o /usr/share/keyrings/mongodb-server-8.0.gpg --dearmor
    echo "deb [ arch=amd64,arm64 signed-by=/usr/share/keyrings/mongodb-server-8.0.gpg ] https://repo.mongodb.org/apt/ubuntu noble/mongodb-org/8.0 multiverse" | sudo tee /etc/apt/sources.list.d/mongodb-org-8.0.list
    sudo apt-get update
    sudo apt-get install -y mongodb-org
    sudo systemctl start mongod
    sudo systemctl enable mongod
    sudo systemctl status mongod
    #-----------------------------------------------------------
    #Install Open5GS
    sudo add-apt-repository ppa:open5gs/latest
    sudo apt update
    sudo apt install -y open5gs

    #-----------------------------------------------------------
    # Download and import the Nodesource GPG key
    sudo apt update
    sudo apt install -y ca-certificates curl gnupg
    sudo mkdir -p /etc/apt/keyrings
    curl -fsSL https://deb.nodesource.com/gpgkey/nodesource-repo.gpg.key | sudo gpg --dearmor -o /etc/apt/keyrings/nodesource.gpg

    NODE_MAJOR=20
    echo "deb [signed-by=/etc/apt/keyrings/nodesource.gpg] https://deb.nodesource.com/node_$NODE_MAJOR.x nodistro main" | sudo tee /etc/apt/sources.list.d/nodesource.list
    #Run Update and Install
    sudo apt update
    sudo apt install nodejs -y

    #-----------------------------------------------------------
    #Install Open5GS WebUI
    curl -fsSL https://open5gs.org/open5gs/assets/webui/install | sudo -E bash -

    sudo systemctl start open5gs-webui
    sudo systemctl enable open5gs-webui

    sudo sed -i "s/'localhost'/'10.0.2.15'/" /usr/lib/node_modules/open5gs/server/index.js # Change localhost
    sudo systemctl daemon-reload
    sudo systemctl restart open5gs-webui.service
    sudo systemctl restart open5gs-webui

    #-----------------------------------------------------------
    #Installation of UERANSIM
    git clone https://github.com/aligungr/UERANSIM
    sudo apt install -y cmake
    sudo apt install -y gcc
    sudo apt install -y g++
    sudo apt install -y libsctp-dev lksctp-tools
    sudo apt install -y iproute2
    sudo snap install cmake --classic
    cd UERANSIM
    make

    #-----------------------------------------------------------
    #Change IP addresses in Open5GS and UERANSIM configuration files
    sudo sed -i '23s/127.0.0.5/10.0.2.15/' /etc/open5gs/amf.yaml # Change AMF IP address
    sudo systemctl restart open5gs-amfd # Restart AMF service to apply changes
    sudo sed -i 's/127.0.0.1/10.0.2.15/' ~/ UERANSIM/config/open5gs-ue.yaml # Change UE IP address
    sudo sed -i 's/127.0.0.1/10.0.2.15/' ~/ UERANSIM/config/open5gs-gnb.yaml # Change gNB IP address
    sudo sed -i 's/127.0.0.5/10.0.2.15/' ~/ UERANSIM/config/open5gs-gnb.yaml # Change gNB AMF IP address
\end{lstlisting}