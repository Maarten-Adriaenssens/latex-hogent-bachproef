\section{\IfLanguageName{dutch}{Vagrantfile}{Vagrantfile}}
\begin{lstlisting}[basicstyle=\small, frame=single, breaklines=true, postbreak=\mbox{\textcolor{red}{$\hookrightarrow$}\space}, escapeinside ={\%,}, escapechar={!}, numbers=left, language=sh, caption=Vagrantfile]
Vagrant.configure("2") do |config|
    config.vm.provider "virtualbox" do |vb|
        # Stel de gedeelde map in om automatisch te mounten
        vb.customize ["sharedfolder", "add", :id, "--name", "vagrant_data", "--hostpath", ".", "--automount"]
    end

    # Server 1
    config.vm.define "conductor" do |cond|
        cond.vm.box = "gusztavvargadr/ubuntu-server"
        cond.vm.box_version = "2404.0.2503"
        cond.vm.network "private_network", ip: "10.0.0.10"
        # Set the host name of the VM
        cond.vm.hostname = "Centrale"
        # Set portforwarding
        cond.vm.network "forwarded_port", guest: 9999, host: 9999, host_ip: "127.0.0.1", id: "open5gs"
        # VirtualBox specific configuration
        cond.vm.provider "virtualbox" do |vb|
            # VirtualBox Display Name
            vb.name = "Centrale"
            # VirtualBox Group
            vb.customize ["modifyvm", :id, "--groups", "/BPoc"]
            # 1GB vRAM
            vb.memory = "8192"
            # 1vCPU
            vb.cpus = "2"
        end
        cond.vm.provision "shell", path: "script-cent.sh"
    end

    config.vm.define "member" do |memb|
        memb.vm.box = "gusztavvargadr/ubuntu-server"
        memb.vm.box_version = "2404.0.2503"
        memb.vm.network "private_network", ip: "10.0.0.11"
        memb.vm.hostname = "ue"
        memb.vm.provider "virtualbox" do |vb2|
            # VirtualBox Display Name
            vb2.name = "UE"
            # VirtualBox Group
            vb2.customize ["modifyvm", :id, "--groups", "/BPoc"]
            # 1GB vRAM
            vb2.memory = "4096"
            # 1vCPU
            vb2.cpus = "2"
        end
        memb.vm.provision "shell", path: "script-ue.sh"
    end
end
\end{lstlisting}

\section{\IfLanguageName{dutch}{Vagrant script voor Conductor VM of 5G Core Server}{Vagrant script for Conductor VM or 5G Core Server}}

\begin{lstlisting}[basicstyle=\small, frame=single, breaklines=true, postbreak=\mbox{\textcolor{red}{$\hookrightarrow$}\space}, escapeinside ={\%,}, escapechar={!}, numbers=left, language=sh, caption=Vagrant script voor Conductor VM of 5G Core Server]
# Script to install Open5GS on Ubuntu 22.04
#Device: Conductur / Centrale
#-----------------------------------------------------------
#Install MongoDB for Open5GS
sudo apt-get install gnupg curl
curl -fsSL https://www.mongodb.org/static/pgp/server-8.0.asc | sudo gpg -o /usr/share/keyrings/mongodb-server-8.0.gpg --dearmor
echo "deb [ arch=amd64,arm64 signed-by=/usr/share/keyrings/mongodb-server-8.0.gpg ] https://repo.mongodb.org/apt/ubuntu noble/mongodb-org/8.0 multiverse" | sudo tee /etc/apt/sources.list.d/mongodb-org-8.0.list
sudo apt-get update
sudo apt-get install -y mongodb-org
sudo systemctl start mongod
sudo systemctl enable mongod
sudo systemctl status mongod
#-----------------------------------------------------------
#Install Open5GS
sudo add-apt-repository ppa:open5gs/latest
sudo apt update
sudo apt install -y open5gs
#-----------------------------------------------------------
# Download and import the Nodesource GPG key
sudo apt update
sudo apt install -y ca-certificates curl gnupg
sudo mkdir -p /etc/apt/keyrings
curl -fsSL https://deb.nodesource.com/gpgkey/nodesource-repo.gpg.key | sudo gpg --dearmor -o /etc/apt/keyrings/nodesource.gpg
NODE_MAJOR=20
echo "deb [signed-by=/etc/apt/keyrings/nodesource.gpg] https://deb.nodesource.com/node_$NODE_MAJOR.x nodistro main" | sudo tee /etc/apt/sources.list.d/nodesource.list
#Run Update and Install
sudo apt update
sudo apt install nodejs -y
#-----------------------------------------------------------
#Install Open5GS WebUI
curl -fsSL https://open5gs.org/open5gs/assets/webui/install | sudo -E bash -
sudo systemctl start open5gs-webui
sudo systemctl enable open5gs-webui
sudo sed -i "s/'localhost'/'10.0.2.15'/" /usr/lib/node_modules/open5gs/server/index.js # Change localhost
sudo systemctl daemon-reload
sudo systemctl restart open5gs-webui.service
sudo systemctl restart open5gs-webui
#-----------------------------------------------------------
#Installation of UERANSIM
git clone https://github.com/aligungr/UERANSIM
sudo apt install -y cmake
sudo apt install -y gcc
sudo apt install -y g++
sudo apt install -y libsctp-dev lksctp-tools
sudo apt install -y iproute2
sudo snap install cmake --classic
cd UERANSIM
make
#-----------------------------------------------------------
#Change IP addresses in Open5GS and UERANSIM configuration files
sudo sed -i '23s/127.0.0.5/10.0.0.10/' /etc/open5gs/amf.yaml # Change AMF IP address
sudo sed -i '20s/127.0.0.7/10.0.0.10/' /etc/open5gs/upf.yaml # Change UPF IP address
sudo systemctl daemon-reload # Reload systemd to recognize changes
sudo systemctl restart open5gs-amfd # Restart AMF service to apply changes
sudo systemctl restart open5gs-upfd # Restart UPF service to apply changes
#-----------------------------------------------------------
#NAT Port Forwarding
sudo sysctl -w net.ipv4.ip_forward=1
sudo iptables -t nat -A POSTROUTING -o eth0 -j MASQUERADE
sudo systemctl stop ufw
sudo iptables -I FORWARD 1 -j ACCEPT
#-----------------------------------------------------------
#End of script
\end{lstlisting}

\section{\IfLanguageName{dutch}{Vagrant script voor Member VM of 5G RAN Server}{Vagrant script for Member VM or 5G RAN Server}}

\begin{lstlisting}[basicstyle=\small, frame=single, breaklines=true, postbreak=\mbox{\textcolor{red}{$\hookrightarrow$}\space}, escapeinside ={\%,}, escapechar={!}, numbers=left, language=sh, caption=Vagrant script voor Member VM of 5G RAN Server]
    # Script to install UERANSIM on Ubuntu 22.04
# Device: Member / UE
#-----------------------------------------------------------
#Installation of UERANSIM
git clone https://github.com/aligungr/UERANSIM
sudo apt install -y cmake
sudo apt install -y gcc
sudo apt install -y g++
sudo apt install -y libsctp-dev lksctp-tools
sudo apt install -y iproute2
sudo snap install cmake --classic
cd UERANSIM
make
#-----------------------------------------------------------
# #Change IP addresses in Open5GS and UERANSIM configuration files
# sudo sed -i '23s/127.0.0.5/10.0.2.15/' /etc/open5gs/amf.yaml # Change AMF IP address
# sudo systemctl restart open5gs-amfd # Restart AMF service to apply changes
sudo sed -i 's/127.0.0.1/10.0.0.11/' ~/UERANSIM/config/open5gs-ue.yaml # Change UE IP address
sudo sed -i 's/127.0.0.1/10.0.0.11/' ~/UERANSIM/config/open5gs-gnb.yaml # Change gNB IP address
sudo sed -i 's/127.0.0.5/10.0.0.10/' ~/UERANSIM/config/open5gs-gnb.yaml # Change gNB AMF IP address
#-----------------------------------------------------------
#End of script
\end{lstlisting}