%%=============================================================================
%% Conclusie
%%=============================================================================

\chapter{Conclusie}%
\label{ch:conclusie}

% TODO: Trek een duidelijke conclusie, in de vorm van een antwoord op de
% onderzoeksvra(a)g(en). Wat was jouw bijdrage aan het onderzoeksdomein en
% hoe biedt dit meerwaarde aan het vakgebied/doelgroep? 
% Reflecteer kritisch over het resultaat. In Engelse teksten wordt deze sectie
% ``Discussion'' genoemd. Had je deze uitkomst verwacht? Zijn er zaken die nog
% niet duidelijk zijn?
% Heeft het onderzoek geleid tot nieuwe vragen die uitnodigen tot verder 
%onderzoek?

% Het doel van deze thesis is om een alternatief te bieden voor het verouderde DECT-systeem. Dit door eerst het DECT-systeem in kaart te brengen met zijn voordelen en nadelen, maar ook de gebreken en mogelijke problemen. Als eenmaal de gebreken gekend zijn, kan er gekeken worden om deze op te vangen met een moderner systeem zoals 5G. Er zijn echter verschillende alternatieven. Een zeer aanneembare verwachting is dat er zal worden gekozen voor een 5G privaat netwerk, omwille van zijn eigenschappen en de afweging tussen de voor- en nadelen. Hoewel er geen officiële standaard is voor 5G-netwerken binnen de gezondheidszorg, zal er een zo objectief mogelijke keuze worden gemaakt met een vergelijking tussen alle mogelijkheden. Een disclaimer is wel noodzakelijk, aangezien het hier gaat om een bachelorproef zal er voornamelijk gekeken worden naar open-source projecten, omwille van budgetten. \\\\
% De Proof-of-Concept zal een simulatie zijn van een kleinschalig, privaat 5G-netwerk. Hoewel kleinschalig, zal het ontworpen zijn met het oog op schaalbaarheid. Het is te verwachten dat de 5G-oplossing een efficiëntere en accuratere manier voor communicatie zal zijn ten opzichte van het DECT-systeem. Verder wordt er een daling verwacht in het voorkomen van alarmmoeheid. Dit komt doordat er verschillende alarmsignalen gebundeld zullen worden. Er zal ook de mogelijkheid zijn om afbeeldingen of berichten te kunnen sturen, waardoor de arts de situatie van de patiënt beter kan inschatten.\\\\
% Op de vraag vanuit Citymesh voor een techno-economische studie, wordt verwacht dat de kost voor het installeren en uitrollen van het private 5G-netwerk hoog zal zijn in vergelijking met het huidige DECT-systeem. Er zal een afweging moeten worden gemaakt ten opzichte van de kost en de mogelijke winst op lange termijn door welzijn van personeel en daling van incidenten, die te wijten zijn aan alarmmoeheid.\\\\
% De volgende stap na deze bachelorproef is de mogelijke integratie van het systeem in de opleidingen binnen het departement Gezondheidszorg van HOGENT, om de studenten een alternatief te kunnen aanbieden ten opzichte van het DECT-systeem.

In dit onderzoek is gezocht naar een antwoord op de vraag: \textit{''Hoe kan de informatie-uitwisseling tussen artsen en verplegend personeel in ziekenhuizen verbeterd worden door het inzetten van nieuwe technologieën ?''}. Hiervoor is er zowel onderzoek uitgevoerd naar een mogelijke andere technologie als een \gls{poc} opgesteld, waarin een privaat 5G netwerk is gesimuleerd en getest. \\\\

Uit het onderzoek blijkt dat het \gls{dect}-systeem een  
%   \item \textit{Wat zijn de kenmerken van het \gls{dect}-systeem?}
%   \item \textit{Wat is het huidige systeem (met \hyphenation{rand-apparatuur}randapparatuur) en wat zijn de nadelen hiervan?}
%   \item \textit{Hoe kan een zorgsituatie worden gesimuleerd?}
%   \item \textit{Welke pogingen zijn al ondernomen om het \gls{dect}-systeem als standaard aan te passen?}
%   \item \textit{Wat zijn de minimumvereisten voor technologieën, om als alternatief te kunnen worden beschouwd?}
%   \item \textit{Welke andere communicatiemogelijkheden zijn er die voldoen aan de minimumvereisten?}
%   \item \textit{Welke mogelijke data integratie mogelijkheden hebben de alternatieven?}
%   \item \textit{Vanaf wanneer is het haalbaar om te implementeren op economisch vlak?}

De conclusie kan worden opgesplitst in een theoretische en een praktische conclusie. 


Vervolgonderzoek wou kunnen kijken naar de omzetting naar praktijk, security implementatie en de mogelijke integratie van het systeem in de opleidingen binnen het departement Gezondheidszorg van HOGENT, om de studenten een alternatief te kunnen aanbieden ten opzichte van het DECT-systeem.



% 1. Restate the Main Research Question or Objective
% Briefly remind the reader of your central research question(s) or objectives.
% Introduction to the Conclusion
% (“This thesis set out to investigate…”)


% 2. Summarize Key Findings
% Highlight the main findings or arguments of your thesis. Focus on the most important results, not every detail.
% Summary of Main Findings
% (“The results show that…”)

% 3. Discuss the Significance and Implications
% Explain why your findings matter. What do they mean for your field? How do they contribute to existing knowledge or practice?
% Implications
% (“These findings suggest that…”)

% 4. Reflect on Limitations
% Acknowledge any limitations in your research. This shows critical thinking and honesty.
% Limitations
% (“However, this study was limited by…”)

% 5. Suggest Recommendations or Future Research
% Offer suggestions for future research or practical recommendations based on your findings.
% Recommendations/Future Research
% (“Further research could explore…”)

% 6. End with a Strong Closing Statement
% Finish with a thought-provoking insight, a call to action, or a statement about the broader impact of your work.
% Final Statement
% (“Ultimately, this thesis demonstrates…”)