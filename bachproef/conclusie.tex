%%=============================================================================
%% Conclusie
%%=============================================================================

\chapter{Conclusie}%
\label{ch:conclusie}

% TODO: Trek een duidelijke conclusie, in de vorm van een antwoord op de
% onderzoeksvra(a)g(en). Wat was jouw bijdrage aan het onderzoeksdomein en
% hoe biedt dit meerwaarde aan het vakgebied/doelgroep? 
% Reflecteer kritisch over het resultaat. In Engelse teksten wordt deze sectie
% ``Discussion'' genoemd. Had je deze uitkomst verwacht? Zijn er zaken die nog
% niet duidelijk zijn?
% Heeft het onderzoek geleid tot nieuwe vragen die uitnodigen tot verder 
%onderzoek?

% Het doel van deze thesis is om een alternatief te bieden voor het verouderde DECT-systeem. Dit door eerst het DECT-systeem in kaart te brengen met zijn voordelen en nadelen, maar ook de gebreken en mogelijke problemen. Als eenmaal de gebreken gekend zijn, kan er gekeken worden om deze op te vangen met een moderner systeem zoals 5G. Er zijn echter verschillende alternatieven. Een zeer aanneembare verwachting is dat er zal worden gekozen voor een 5G privaat netwerk, omwille van zijn eigenschappen en de afweging tussen de voor- en nadelen. Hoewel er geen officiële standaard is voor 5G-netwerken binnen de gezondheidszorg, zal er een zo objectief mogelijke keuze worden gemaakt met een vergelijking tussen alle mogelijkheden. Een disclaimer is wel noodzakelijk, aangezien het hier gaat om een bachelorproef zal er voornamelijk gekeken worden naar open-source projecten, omwille van budgetten. \\\\
% De Proof-of-Concept zal een simulatie zijn van een kleinschalig, privaat 5G-netwerk. Hoewel kleinschalig, zal het ontworpen zijn met het oog op schaalbaarheid. Het is te verwachten dat de 5G-oplossing een efficiëntere en accuratere manier voor communicatie zal zijn ten opzichte van het DECT-systeem. Verder wordt er een daling verwacht in het voorkomen van alarmmoeheid. Dit komt doordat er verschillende alarmsignalen gebundeld zullen worden. Er zal ook de mogelijkheid zijn om afbeeldingen of berichten te kunnen sturen, waardoor de arts de situatie van de patiënt beter kan inschatten.\\\\
% Op de vraag vanuit Citymesh voor een techno-economische studie, wordt verwacht dat de kost voor het installeren en uitrollen van het private 5G-netwerk hoog zal zijn in vergelijking met het huidige DECT-systeem. Er zal een afweging moeten worden gemaakt ten opzichte van de kost en de mogelijke winst op lange termijn door welzijn van personeel en daling van incidenten, die te wijten zijn aan alarmmoeheid.\\\\
% De volgende stap na deze bachelorproef is de mogelijke integratie van het systeem in de opleidingen binnen het departement Gezondheidszorg van HOGENT, om de studenten een alternatief te kunnen aanbieden ten opzichte van het DECT-systeem.

In dit onderzoek is gezocht naar een antwoord op de vraag: \textit{''Hoe kan de informatie-uitwisseling tussen artsen en verplegend personeel in ziekenhuizen verbeterd worden door het inzetten van nieuwe technologieën ?''}. Hiervoor is er zowel onderzoek uitgevoerd naar een mogelijke andere technologie als een \gls{poc} opgesteld, waarin een privaat 5G netwerk is gesimuleerd en getest. \\

Uit het onderzoek blijkt dat het \gls{dect}-systeem een  hoge betrouwbaarheid, lage energieconsumptie en een eigen frequentieband met 120 duplex radio kanalen. Echter het is verouderd,  beperkte capaciteit verkeer en slechte integratie. \\

Na het formuleren van de minimumvereisten aan de hand van de MoSCoW methode, zijn volgende eigenschappen noodzakelijk om over een acceptabele vervanging te spreken: Wetgeving conform, gesloten systeem, stabiel en segregatie mogelijkheden. Wel is het noodzakelijk om in het achterhoofd te houden dat het om een vervanging/upgrade gaat dus technologieën moeten minsten gelijkstaan met het \gls{dect}-systeem.\\
Uit de analyse van verschillende technologieën blijkt dat \gls{voip}, \gls{ule} en private 5G-netwerken voldoen aan deze eisen. Deze alternatieven bieden bovendien betere mogelijkheden voor data integratie.\\
Economische haalbaarheid is sterk afhankelijk van de schaal van de implementatie en of er wordt gebruikgemaakt van een gefaseerd project. Verder is de kost groter dan de infrastructuur alleen. Er moet dus niet alleen rekening worden gehouden met de \gls{capex}, maar ook met de \gls{opex}. 
Ook bleek uit de \gls{poc} dat deze een deel van zijn packets al encrypteerd en encapsuleerd. Dit zorgt op zich al voor een eerste laag securtiy die vereist is van een vervangingstechnologie.

De eindconclusie kan dus worden opgesplitst in twee delen: een theoretische en een praktische eindconclusie.\\
Theoretisch is een privaat 5G netwerk het effectiefst en het meest toekomstgericht. Dit omdat de data integratie het hoogste ligt en door gebruik te maken van slicing kan met optimale segregatie hebben van communicatiekanalen, al dan niet met prioriteiten. De praktijk schetst natuurlijk ook een beeld, de hoge kost voor het installeren van zo een netwerk kan verschillende ziekenhuizen. Dit omdat men verder kan werken met de huidige installatie mits wat aanpassingen. Een andere optie is het overschakelen naar \gls{voip}, dit is het alternatief met de minste voorkeur omwille van de security.\\

Vervolgonderzoek wou kunnen kijken naar de omzetting naar praktijk, security implementatie en de mogelijke integratie van het systeem in de opleidingen binnen het departement Gezondheidszorg van HOGENT, om de studenten een alternatief te kunnen aanbieden ten opzichte van het DECT-systeem.



%   \item \textit{Wat zijn de kenmerken van het \gls{dect}-systeem?} ✓
%   \item \textit{Wat is het huidige systeem (met \hyphenation{rand-apparatuur}randapparatuur) en wat zijn de nadelen hiervan?}✓
%   \item \textit{Hoe kan een zorgsituatie worden gesimuleerd?}
%   \item \textit{Welke pogingen zijn al ondernomen om het \gls{dect}-systeem als standaard aan te passen?}✓
%   \item \textit{Wat zijn de minimumvereisten voor technologieën, om als alternatief te kunnen worden beschouwd?}✓
%   \item \textit{Welke andere communicatiemogelijkheden zijn er die voldoen aan de minimumvereisten?}✓
%   \item \textit{Welke mogelijke data integratie mogelijkheden hebben de alternatieven?}✓
%   \item \textit{Vanaf wanneer is het haalbaar om te implementeren op economisch vlak?}✓
