%%=============================================================================
%% Conclusie
%%=============================================================================

\chapter{Conclusie}%
\label{ch:conclusie}

% TODO: Trek een duidelijke conclusie, in de vorm van een antwoord op de
% onderzoeksvra(a)g(en). Wat was jouw bijdrage aan het onderzoeksdomein en
% hoe biedt dit meerwaarde aan het vakgebied/doelgroep? 
% Reflecteer kritisch over het resultaat. In Engelse teksten wordt deze sectie
% ``Discussion'' genoemd. Had je deze uitkomst verwacht? Zijn er zaken die nog
% niet duidelijk zijn?
% Heeft het onderzoek geleid tot nieuwe vragen die uitnodigen tot verder 
%onderzoek?

% Het doel van deze thesis is om een alternatief te bieden voor het verouderde DECT-systeem. Dit door eerst het DECT-systeem in kaart te brengen met zijn voordelen en nadelen, maar ook de gebreken en mogelijke problemen. Als eenmaal de gebreken gekend zijn, kan er gekeken worden om deze op te vangen met een moderner systeem zoals 5G. Er zijn echter verschillende alternatieven. Een zeer aanneembare verwachting is dat er zal worden gekozen voor een 5G privaat netwerk, omwille van zijn eigenschappen en de afweging tussen de voor- en nadelen. Hoewel er geen officiële standaard is voor 5G-netwerken binnen de gezondheidszorg, zal er een zo objectief mogelijke keuze worden gemaakt met een vergelijking tussen alle mogelijkheden. Een disclaimer is wel noodzakelijk, aangezien het hier gaat om een bachelorproef zal er voornamelijk gekeken worden naar open-source projecten, omwille van budgetten. \\\\
% De Proof-of-Concept zal een simulatie zijn van een kleinschalig, privaat 5G-netwerk. Hoewel kleinschalig, zal het ontworpen zijn met het oog op schaalbaarheid. Het is te verwachten dat de 5G-oplossing een efficiëntere en accuratere manier voor communicatie zal zijn ten opzichte van het DECT-systeem. Verder wordt er een daling verwacht in het voorkomen van alarmmoeheid. Dit komt doordat er verschillende alarmsignalen gebundeld zullen worden. Er zal ook de mogelijkheid zijn om afbeeldingen of berichten te kunnen sturen, waardoor de arts de situatie van de patiënt beter kan inschatten.\\\\
% Op de vraag vanuit Citymesh voor een techno-economische studie, wordt verwacht dat de kost voor het installeren en uitrollen van het private 5G-netwerk hoog zal zijn in vergelijking met het huidige DECT-systeem. Er zal een afweging moeten worden gemaakt ten opzichte van de kost en de mogelijke winst op lange termijn door welzijn van personeel en daling van incidenten, die te wijten zijn aan alarmmoeheid.\\\\
% De volgende stap na deze bachelorproef is de mogelijke integratie van het systeem in de opleidingen binnen het departement Gezondheidszorg van HOGENT, om de studenten een alternatief te kunnen aanbieden ten opzichte van het DECT-systeem.



