\chapter{\IfLanguageName{dutch}{Minimum Requirements}{Minimum Requirements}}%
\label{ch:minreq}

Zoals aangehaald in de methodologie, is er een noodzaak voor een duidelijke oplijsting van de mimimum Requirements. Op deze manier kan er een arbitraire keuze worden gemaakt tussen alle technologieën. Verder zal deze oplijsting ook worden gebruikt om de \acrshort{poc} af te toetsen tegen de \acrshort{mvp}. Dit is noodzakelijk om tegen eht einde van de bachelorproef een praktische realisatie te kunnen opleveren die, zelfs in de meest basis versie een goed alternatief kan bieden voor het huide \acrshort{dect}.

\begin{enumerate}
    \item Betrouwbaarheid en Beschikbaarheid
    \item Beveiliging en Privacy
    \item \acrofull{qos}
    \item Interoperabiliteit
    \item Mobilitetit en dekking
    \item Schaalbaarheid
    \item Gebruiksvriendelijkheid
    \item Integratie met (huidige) klinische workflows
\end{enumerate}

\subsection{\IfLanguageName{dutch}{Betrouwbaarheid en Beschikbaarheid}{Reliability and Availability}}

\subsection{\IfLanguageName{dutch}{Betrouwbaarheid en Beschikbaarheid}{Reliability and Availability}}

\subsection{\IfLanguageName{dutch}{Betrouwbaarheid en Beschikbaarheid}{Reliability and Availability}}

\subsection{\IfLanguageName{dutch}{Betrouwbaarheid en Beschikbaarheid}{Reliability and Availability}}

\subsection{\IfLanguageName{dutch}{Betrouwbaarheid en Beschikbaarheid}{Reliability and Availability}}

\subsection{\IfLanguageName{dutch}{Betrouwbaarheid en Beschikbaarheid}{Reliability and Availability}}

\subsection{\IfLanguageName{dutch}{Betrouwbaarheid en Beschikbaarheid}{Reliability and Availability}}

\subsection{\IfLanguageName{dutch}{Betrouwbaarheid en Beschikbaarheid}{Reliability and Availability}}