\chapter{\IfLanguageName{dutch}{Minimum Requirements}{Minimum Requirements}}%
\label{ch:minReq}

Zoals aangehaald in de methodologie, is er een noodzaak voor een duidelijke oplijsting van de mimimum Requirements. Op deze manier kan er een arbitraire keuze worden gemaakt tussen alle technologieën. Verder zal deze oplijsting ook worden gebruikt om de \acrshort{poc} af te toetsen tegen de \acrshort{mvp}. Dit is noodzakelijk om tegen eht einde van de bachelorproef een praktische realisatie te kunnen opleveren die, zelfs in de meest basis versie een goed alternatief kan bieden voor het huide \acrshort{dect}.

De requirements zullen worden opgesteld volgens het Moscow methode.

Volgens \textcite{Coiera2006} zijn volgende aspecten van een ocmuunicatiesysteem voor ziekenhuizen van uiterst belang:

\begin{enumerate}
    \item Betrouwbaarheid en Beschikbaarheid
    \item Beveiliging en Privacy
    \item \acrfull{qos}
    \item Interoperabiliteit
    \item Mobilitetit en Dekking
    \item Schaalbaarheid
    \item Gebruiksvriendelijkheid
    \item Integratie met (huidige) klinische workflows 
\end{enumerate}

Verder wordt er met het oog op de toekomst ook extra requirements toegevoegd. Deze zijn de volgende:

\begin{enumerate}
    \item Beschikbaarheid van Data
    \item Prioritisering
\end{enumerate}

\textbf{DIT HOOFDSTUK ZAL WORDEN AANGEVULD TEGEN DE 2DE DEADLINE (SEMESTER 2)}

\section{\IfLanguageName{dutch}{Betrouwbaarheid en Beschikbaarheid}{Reliability and Availability}}
\label{sec:betrouwbaarheid-en-beschikbaarheid}
%TODO:


\section{\IfLanguageName{dutch}{Beveiliging en Privacy}{Security and Privacy}}
\label{sec:beveiliging-en-privacy}
%TODO:



\section{\IfLanguageName{dutch}{Quality of Service}{Quality of Service}}
\label{sec:quality-of-service}
%TODO:



\section{\IfLanguageName{dutch}{Interoperabiliteit}{Interoperability}}
\label{sec:interoperabiliteit}
%TODO:



\section{\IfLanguageName{dutch}{Mobilitetit en Dekking}{Mobility and Coverage}}
\label{sec:mobilitetit-en-dekking}
%TODO:



\section{\IfLanguageName{dutch}{Schaalbaarheid}{Scalability}}
\label{sec:schaalbaarheid}
%TODO:



\section{\IfLanguageName{dutch}{Gebruiksvriendelijkheid}{User-friendly Interface}}
\label{sec:gebruiksvriendelijkheid}
%TODO:



\section{\IfLanguageName{dutch}{Integratie met (huidige) klinische workflows}{Integration with Clinical Workflows}}
\label{sec:integratie-met-klinische-workflows}
%TODO:



\section{\IfLanguageName{dutch}{Vergelijken van technologieën}{Comparing Technologies}}
\label{sec:vergelijken-van-technologieën}
%TODO:



Deze cijfers zijn niet accuraat en moeten nog aangepast worden:

\begin{table}[h!]
    \begin{center}
    \caption{Technologieën overzicht met minimumvereisten}
    \label{tab:Vergelijken van technologieën}
    \begin{tabular}{|l|c|c|c|c|}
    & \textbf{\gls{lte} Advanced} & \textbf{5G} & \textbf{\gls{ule}} & \textbf{\gls{voip}}\\
    \hline
    Betrouwbaarheid en Beschikbaarheid & - & - & - & -\\
    Beveiliging en Privacy & - & - & - & - \\
    \acrfull{qos} & - & - & - & -\\
    Interoperabiliteit & - & - & - & -\\
    Mobilitetit en Dekking & - & - & - & -\\
    Schaalbaarheid & - & - & - & -\\
    Gebruiksvriendelijkheid & - & - & - & -\\
    Integratie met (huidige) klinische workflows  & - & - & - & -\\
    \hline
    \textbf{Totale score}  & - & - & - & -\\
    \end{tabular}
    \end{center}
    \end{table}
