%%=============================================================================
%% Samenvatting
%%=============================================================================

% TODO: De "abstract" of samenvatting is een kernachtige (~ 1 blz. voor een
% thesis) synthese van het document.
%
% Een goede abstract biedt een kernachtig antwoord op volgende vragen:
%
% 1. Waarover gaat de bachelorproef?
% 2. Waarom heb je er over geschreven?
% 3. Hoe heb je het onderzoek uitgevoerd?
% 4. Wat waren de resultaten? Wat blijkt uit je onderzoek?
% 5. Wat betekenen je resultaten? Wat is de relevantie voor het werkveld?
%
% Daarom bestaat een abstract uit volgende componenten:
%
% - inleiding + kaderen thema
% - probleemstelling
% - (centrale) onderzoeksvraag
% - onderzoeksdoelstelling
% - methodologie
% - resultaten (beperk tot de belangrijkste, relevant voor de onderzoeksvraag)
% - conclusies, aanbevelingen, beperkingen
%
% LET OP! Een samenvatting is GEEN voorwoord!

%%---------- Nederlandse samenvatting -----------------------------------------
%
% TODO: Als je je bachelorproef in het Engels schrijft, moet je eerst een
% Nederlandse samenvatting invoegen. Haal daarvoor onderstaande code uit
% commentaar.
% Wie zijn bachelorproef in het Nederlands schrijft, kan dit negeren, de inhoud
% wordt niet in het document ingevoegd.

\IfLanguageName{english}{%
\selectlanguage{dutch}
\chapter*{Samenvatting}
\lipsum[1-4]
\selectlanguage{english}
}{}

%%---------- Samenvatting -----------------------------------------------------
% De samenvatting in de hoofdtaal van het document

\chapter*{\IfLanguageName{dutch}{Samenvatting}{Abstract}}

Communicatie is een heel belangrijk onderdeel van een goede werking van een ziekenhuis. Toch is het meest gebruikte communicatie systeem in ziekehuizen sinds 1993 niet veranderd van principe. Dit pricipe/systeem is het DECT-systeem. Dit onderzoek focust zich op het onderzoeken van de mogelijke technologieën die kunnen worden ingezet ingezet om de communicatie aan te pakken in de gezondheidszorg. De reden van dit onderzoek is om een kwaliteitsvol alternatief te bieden voor de vervanging van het DECT-systeem. Een reden voor deze vervanging is dat deze aanleiding geeft tot alarmmoeheid.
Volgens \textcite{Ferrara2023} kan alarmmoeheid worden omschreven als ''een overprikkeling of zintuiglijke overbelasting, in staat tot het veroorzaken van gevoelloosheid ten opzichte van alarmen door een te groot aantal alarmen die foutief of klinisch onbelangrijk zijn.''.\\ Eerst zal er breed worden gekeken naar alle mogelijke innovaties, nadien wordt er gefocust op een toepassing van een privaat 5G-netwerk. 
Vervolgens worden er twee proof-of-concepts (PoC) gemaakt. De eerste is een testomgeving, lokaal op de laptop van de student. De tweede PoC is een praktische realisatie; deze wordt ontwikkeld op kleine schaal, maar met uitbreidingsmogelijkheden. Tenslotte wordt er ook onderzocht wat de kost is van een 5G-deployment. Dit wordt verwerkt zodat er een overzicht is van wanneer een ziekenhuis het zou overwegen om de kosten te kunnen verantwoorden.\\
Er wordt verwacht dat de PoC's beiden worden gerealiseerd en een overzicht wordt opgesteld met betrekking tot de kost voor deployment en support. De kost om een privaat 5G-netwerk te creëren is hoog, het is dus een logische redenering om te concluderen dat enkel bij grote ziekenhuizen deze kost vaak verantwoord kan worden. Echter, er zal een afweging moeten worden gemaakt ten opzichte van de productiviteitsboost door de alarmmoeheidvermindering.
