%---------- Inleiding ---------------------------------------------------------

% TODO: Is dit voorstel gebaseerd op een paper van Research Methods die je
% vorig jaar hebt ingediend? Heb je daarbij eventueel samengewerkt met een
% andere student?
% Zo ja, haal dan de tekst hieronder uit commentaar en pas aan.

%\paragraph{Opmerking}

% Dit voorstel is gebaseerd op het onderzoeksvoorstel dat werd geschreven in het
% kader van het vak Research Methods dat ik (vorig/dit) academiejaar heb
% uitgewerkt (met medesturent VOORNAAM NAAM als mede-auteur).
% 

\section{Inleiding}%
\label{sec:inleiding}

\subsection{Probleemstelling}

In de zorgindustrie wordt er steeds meer gesproken over 'alarmmoeheid'. Volgens \textcite{Ferrara2023} kan alarmmoeheid worden omschreven als "een overprikkeling of zintuiglijke overbelasting, in staat tot het veroorzaken van gevoelloosheid ten opzichte van alarmen door een te groot aantal alarmen die foutief of klinisch onbelangrijk zijn." De ernst van de gevolgen van deze moeheid is echter verontrustend. Zo stelt \textcite{Ferrara2023} dat de gedragsmatige reactie van het verplegend personeel, die leidt en een hoge alarmmoeheid, ongepast is. Deze reacties variëren tussen het geluid van het alarmsignaal aanpassen tot volledige uitschakeling van het alarm. Wat dus een mogelijk gevaar voor de patiënt  inhoudt. Deze stelling wordt bevestigd door het onderzoek van \textcite{Casey2018}. Deze stelt dat 90,2 \%, van het geïnterviewde verplegend personeel, foutieve of klinisch onbelangrijke alarmen frequent waren, patiëntenzorg verstoorde en het vertrouwen in het alarmsysteem schaadde. Verder zou tot 80,5\%, van het geïnterviewde  verplegend personeel, het alarmsysteem soms deactiveren.
\\\\
Twee van de onderdelen van dit alarm-/\\monitoringsysteem zijn het DECT-systeem en het BELET-systeem. Het DECT-systeem is het audiosysteem dat gebruikmaakt van een mobiel toestel om binnen een bedrijf, in deze situatie een ziekenhuis, tussen diensten te communiceren. 

In tegenstelling tot het DECT-systeem, ligt het BELET-systeem volledig bij de patiënt. Deze is in bezit van een alarmknop waarmee hulp kan worden opgeroepen.

Het is de menselijke factor in het alarmsysteem waar de alarmmoeheid en andere factoren op kan inspelen en de hulpverlening kan bemoeilijken. Dit wordt duidelijk met een situatieschets van het gebruik van het DECT-systeem:

\begin{itemize}
  \item patiënt bedient alarmknop
  \item Verplegend personeel wordt verwittigd via het BELET-systeem
  \item Verplegend personeel reageert op alarm en bezoekt de patiënt
  \item Patiënt heeft een situatie die verslechtert en verplegend personeel wil de dokter inschakelen
  \item Verplegend personeel legt mondeling het probleem uit aan de dokter
  \item Dokter maakt oordeel
\end{itemize}

Het is de voorlaatste stap, waar één probleem optreedt. De dokter steunt volledig op de beschrijving van het verplegend personeel. Dit betekent dus ook dat, eenmaal de conversatie beëindigd is, er geen naslagwerk is om op terug te vallen. Verder is het DECT-systeem verouderd en biedt het geen integratiemogelijkheid van monitoring toestellen.

\subsection{Bachelorproef}

Deze bachelorproef thesis zal zich focussen op het onderzoeken van welke innovaties een waardevol alternatief kunnen zijn voor het DECT-systeem.Dit onderzoek is gericht op IT-personeel in gezondheidszorginstituten.\\\\
Het primaire doel is om een antwoord te formuleren op de vraag \textit{''Welke innovaties kunnen een waardevolle vervanging zijn van het DECT-systeem?''}. Een van de pistes die wordt onderzocht is \textit{''Hoe kan een 5G-netwerk een oplossing bieden?''}. Een factor voor het verlagen van de alarmmoeheid is het bundelen van alarmsignalen in het 5G-netwerk. Het 5G-netwerk biedt ook extra communicatiemethodes, naast het audiosysteem zoals bij DECT, zoals berichten, foto's , \dots . Op deze manier kan de dokter, stel hij is onbereikbaar omwille van andere interventie, later alles nakijken zonder verder gestoord te worden. \\\\
Het secundaire doel is het opstellen van een proof of concept. De eerste vraag van de proof of concept is \textit{''Hoe ziet de 5G-topologie er uit?''}. Deze wordt in samenwerking met Citymesh opgesteld en met 360° Zorglab getest. Het 360° Zorglab is een project van Hogent. De kern van het Zorglab is het creëren van een wisselwerking tussen disciplines enerzijds en tussen onderzoek, dienstverlening en onderwijs anderzijds. Toekomstige gezondheids- en welzijnsmedewerkers kunnen hier in een veilige omgeving niet-technische skills aanleren in een interprofessionele context. \autocite{HOGENT2024} Het doel van deze samenwerking is een snelle wisselwerking te kunnen creëren om zo meerdere malen de testfases te kunnen doorlopen. Vervolgens wordt er terugkoppeling gemaakt, zodat er een antwoord is op \textit{''Hoe kan de proof of concept voor de simulatie worden aangepast om in realiteit te kunnen gebruiken?''} naar de verschillen tussen deze simulatie en de realiteit. Op deze manier kan er een overzicht worden gemaak.\\\\
Het tertiaire doel is om een techno-economische studie uit te voeren op een privaat 5G-netwerk in de gezondheidszorg. Op deze manier wordt het antwoord op de vraag \textit{''Vanaf wanneer heeft 5G een commercieel en technologisch voordeel in een ziekenhuis?''} in kaart gebracht. Hierbij worden de voor- en nadelen van elke technologie/ innovatie opgelijst en met elkaar vergeleken, samen met de totale kost die hieraan verbondenis .

%---------- Stand van zaken ---------------------------------------------------

\section{Literatuurstudie}%
\label{sec:literatuurstudie}

De huidige stand van zaken kan worden opgesplitst in 4 delen. Het eerste deel is een situatieschets van de huidige complete set-up, met al zijn gebreken en voordelen. Hierna volgt een uiteenzetting van 5G-frameworks. Hier gaat het niet alleen over algemene frameworks en bemerkingen over 5G, maar ook over meer specifieke gezondheidszorgframeworks van 5G. Nadien wordt er dieper gekeken naar de connectie tussen gezondheidszorg en 5G, en hun interconnecties. Deze interconnecties bevatten onder andere monitoring integratie. Tenslotte is er een korte toelichting op de cybersecurityverplichtingen in het kader van deze bachelorproef.

\subsection{DECT-systeem}
Het DECT-systeem staat voor Digital Enhanced Cordless Telecommunications; dit is een standaard in de EU sinds 1993. De meest gebruikte situatie is waar er meerdere gebruikers zijn voor een draadloze communicatie in werkomgevingen. De voornaamste reden voor gebruik van het DECT-\\draadloos systeem is dat het een grote dichtheid van veel gebruikers aankan. In vergelijking met andere mobiele communicatiesystemen werkt het niet buiten de werkomgeving. \autocite{Welinder1997} Na onderzoek van \textcite{Welinder1997} bleek dat de interferentie van het DECT-systeem op medisch gereedschap 11 procent was. De interferentie valt echter volledig weg bij een afstand van 0,5m tussen het gereedschap en het DECT toestel.


Howel het DECT-systeem een standaard is sinds 1993, toch zijn er verschillende innovaties toegepast op dit systeem, maar deze worden als een apart systeem gezien. 

\subsection{5G}
5G is een stap in de evolutie van het mobiele netwerk, opvolger van 4G. In deze bachelorproef gaat het om het private 5G-netwerk. Volgens \textcite{wen2021private} zijn er eigenschappen van private 5G die nauw aansluiten met die van DECT-systemen. De eerste eigenschap wijst op de hoge apparatuur dichtheid, maar bij 5G gaat deze ook nog eens gepaard met hoge throughput. Dit zorgt voor integratie van externe devices, zoals sensoren, camera's, etc. Verdergaand op de gelijkenissen in eigenschappen heeft 5G ook voordelen ten opzichte van DECT-systemen volgens \textcite{wen2021private}. Zo is er een hoge betrouwbaarheid met een lage latency.\\ De tweede eigenschap van het privaat netwerk is de flexibiliteit en voorspelbaarheid van Quality of Service (QoS). Daarnaast zijn er verschillende architecturen die men kan implementeren om een privaat 5G-netwerk op te zetten. De eerste methode is een stand-alone deployment. Hierbij wordt een privaat netwerk opgezet, waarbij alle netwerk functies van het netwerk zijn gelimiteerd binnen een logische perimeter bestaande uit vooraf gedefinieerde regio's. De andere methode is een public netwerk geintegreerd deployment. Deze architectuur kan worden opgesplitst in verschillende types, afhankelijk van de gradatie van integratie. Deze 4 types zijn de volgende:

\begin{itemize}
  \item O-RAN (Open Radio Access Network)
  \item Gedeelde RAN (Radio Access Network)
  \item Gedeelde RAN en controle vlak
  \item Gehost bij het Publieke Netwerk
\end{itemize}



\textcite{wen2021private} vermeldt ook dat er naast architectuur een keuze moet worden gemaakt voor het spectrum. Hier zijn ook opnieuw 3 keuzes: een dedicated privaat spectrum, een erkend spectrum en een niet-erkend spectrum. 
5G heeft een aantal noodzakelijke technologieën nodig in het netwerk. Een van deze technologieën is network slicing. Hierbij wordt er 'een netwerk in een netwerk' gemaakt door het fysieke netwerk op te splitsen in meerdere logische netwerken. Om aan network slicing te kunnen doen is er een nood aan netwerkvirtualisatie. Het slicing zelf kan worden opgedeeld in 3 lagen: Infrastructuurlaag, Network-slice laag en Onderhoudslaag. De levensloop van het slicen van een netwerk verloopt volgens de volgende 4 fases \autocite{wen2021private}:

\begin{enumerate}
  \item Voorbereiding
  \item In dienst stellen
  \item Gebruik
  \item Ontmanteling
\end{enumerate}



\subsection{Connecties tussen 5G en gezondheidszorg}

Het derde luik van de literatuurstudie gaat dieper in op de connecties tussen 5G en gezondheidszorg, en hoe deze worden bereikt aan de hand van bestaande frameworks. Er zijn verschillende mogelijkheden, maar met de focus op een toegankelijke methode zal er voornamelijk gefocust worden op open-source frameworks. Zo lijst \textcite{Eswaran2022} de verschillende variaties op een open-source 5G-framework voor private 5G-netwerken op:

\begin{itemize}
  \item Magma
  \item 5G OpenRAN
  \item ONF's Aether platform
  \item ETSI OSM
  \item srsRAN
  \item OpenAirInterface UE
\end{itemize}

Volgens \textcite{Open5GS2024} is Open5G 'een geavanceerd open-source project ontworpen voor het bouwen en onderhouden van je eigen NR/LTE mobiele netwerk. Open5G biedt een robuuste oplossing voor het configureren van zowel 5G (NR) als LTE (Evolved) netwerken.' \\ Verder zijn er andere concepten zoals ORAN waar onder andere het OpenCare5G-netwerk op gebaseerd is. In dit framework wordt Open RAN in een privaat netwerk gebruikt voor digitale gezondheidsapplicaties. Zo hebben \textcite{de2023opencare5g} onderzocht hoe dit framework kan gebruikt worden om gezondheidsonderzoeken te doen met draagbare ultrasone gereedschappen op verschillende locaties. Om deze onderzoeken te kunnen delen met artsen, werd er een lokaal privaat netwerk opgezet. Zo gebruikten ze een 5G xHaul ORAN Operations, Administration en Maintenance (OAM) privaat netwerk. Er wordt gekozen om een top-down systeem te gebruiken. Dit is een georganiseerde manier om een netwerk project te ontwikkelen. Dit komt doordat men kan steunen op het OSI-model en het 5G-RAN protocol layer model. Dit zal ook gebruikt worden bij de methodologie. Zo zal er een analyse zijn van elke laag van de 5G-architectuur. Beginnend met de service laag, hier komen de verzoeken van de applicatie vandaan. De architectuur eindigt met de infrastructuurlaag. \autocite{de2023opencare5g}


\subsection{Security}
In samenspraak met de co-promotor is er geopteerd om cybersecurity niet actief te onderzoeken in deze bachelorproef thesis. Echter, het is wel noodzakelijk om te weten dat er verschillende verplichtingen zijn in België en de Europese Unie. Hieronder een verduidelijking als context.
\\
Sinds 27 april 2016 is de verordening 2016/679 van toepassing. Deze start de GDPR (General Data Protection Regulation) in de Europese Unie. De verordening 2016/679 vermeldt dat "regels worden vastgelegd betreffende de bescherming van natuurlijke personen in verband met de verwerking van persoonsgegevens en het vrije verkeer van persoonsgegevens. Het beschermt de grondrechten en de fundamentele vrijheden van natuurlijke personen, met name hun recht op bescherming van persoonsgegeven."\\ (Verordening (EU) 2016/679 van het Europees Parlement en de Raad van 27 april 2016) %\autocite{gdpr2016} 
\\\\
Vervolgens is er ook het NIS2, een Belgische wet die bedrijven verplichtingen oplegt op het vlak van cybersecurity. "De wet van 26 april 2024 tot vaststelling van een kader voor de cyberbeveiliging van netwerk- en informatiesystemen van algemeen belang voor de openbare veiligheid (de "NIS2-wet") zet de EU-richtlijn 2022/2555 van het Europees Parlement en de Raad van 14 december 2022 (de "NIS2-richtlijn") om." \\
(Wet tot vaststelling van een kader voor de cyberbeveiliging van netwerk- en informatiesystemen van algemeen belang voor de openbare veiligheid van 26 april 2024) %\autocite{Belgium2024}
\\\\
Tenslotte is er een nieuwer initiatief genaamd EHDS (European Healthcare Data Space). Dit initiatief heeft het volgende doel: "De algemene doelstelling is te waarborgen dat natuurlijke personen in de EU in de praktijk meer zeggenschap over hun elektronische gezondheidsgegevens hebben."\\ (Voorstel (EU) COM/2022/197 van het Europees Parlement en de Raad van 3 mei 2022) %\autocite{EHDS2022}

% Voor literatuurverwijzingen zijn er twee belangrijke commando's:
% \autocite{KEY} => (Auteur, jaartal) Gebruik dit als de naam van de auteur
%   geen onderdeel is van de zin.
% \textcite{KEY} => Auteur (jaartal)  Gebruik dit als de auteursnaam wel een
%   functie heeft in de zin (bv. ``Uit onderzoek door Doll & Hill (1954) bleek
%   ...'')

%---------- Methodologie ------------------------------------------------------
\section{Methodologie}%
\label{sec:methodologie}

% Alvorens de methodologie volledig wordt besproken; is er een belangrijke disclaimer noodzakelijk te vermelden.
% In samenspraak met de co-promotor is er geopteerd om cybersecurity niet actief te onderzoeken in deze bachelorproef thesis. Echter, het is wel noodzakelijk om te weten dat er verschillende verplichtingen zijn in België en de Europese Unie. In de literatuurstudie is er wel een kort naslagwerk van de verplichtingen ten opzichte van cybersecurity.
% \\\\
Tijdens het verloop van de bachelorproef zullen er antwoorden worden vergaard op verschillende deelvragen die helpen de onderzoeksvraag te beantwoorden. Zo zal elke deelvraag deel uitmaken van een onderdeel van de methodologie.
\\\\
De bachelorproef verloopt in de volgende fasen:

\begin{enumerate}
  \item Research / Voorbereiding
  \item In dienst stellen / Setup
  \item Gebruik en testen
  \item Realisatiefactor bbepaling
  \item Conclusie
\end{enumerate}

\subsection{Research / Voorbereidingfase}
De eerste vraag die moet gesteld worden is: \textit{''Welke innovaties kunnen een waardevolle vervanging zijn van het DECT-systeem?''}.\\ 
Hiermee wordt de onderzoeksfase gestart. De eerste stap van de onderzoeksfase is het in kaart brengen van het huidige DECT-systeem. Dit zal bekeken worden vanuit een algemeen overzicht. Het doel is om een topologisch schema en een duidelijk overzicht te hebben van de werking van een DECT-systeem.\\
Vervolgens wordt er een opvolgvraag gesteld: \textit{''Hoe vervangen we het DECT-systeem?''}\\ Deze vraag kan enkel beantwoord worden na een kort onderzoek in de verschillende frameworks, architecturen. Als besluit van dit onderzoek zal er een vergelijkende studie zijn. Hieruit wordt het best passende framework gekozen.
Als vervolg op de keuze van het framework zullen de technische specificaties worden vastgelegd en opgelijst. Dit is een noodzakelijke stap om in de conclusie op het einde een duidelijk beeld te kunnen krijgen van mogelijke upsizing.\\
De volgende deelvraag is: \textit{''Hoe kan een 5G-netwerk een oplossing bieden?''}.\\ Met deze vraag wordt er dieper onderzoek gedaan naar de huidige 5G-systemen/-frameworks. De laatste deelvraag van de research-/voorbereidingsfase is: \textit{''Hoe moet het 5G-netwerk eruitzien, rekening houdend met framework en scaling?''}\\ Hoewel dit de laatste stap is in de voorbereidende fase, is deze cruciaal voor alle verdere fases. Hier wordt alle informatie vergaard in de researchfase en in een netwerkschema/-topologie verwerkt. Dit zal als leidraad worden gebruikt in het verdere verloop van de bachelorproef.
\subsection{In dienst stellen / Setup-fase}
Deze fase bestaat uit het omzetten van de verworven theorie naar een proof-of-concept (PoC). Deze bachelorproef zal bestaan uit 2 PoC's. De eerste zal een simulatie zijn die lokaal op de pc werkt. Het tweede deel is een praktische realisatie, met de artikelen opgelijst in de technische specificaties. 
\subsection{Gebruik en Testfase}
Het doel van deze fase is het antwoorden op de vraag: \textit{''Wat kan er fout gaan en hoe kan dit worden vermeden?''} De gebruik- en testfase verloopt net zoals de setupfase in twee delen. Het eerste deel is het opstellen van een eenduidig en gedetailleerd testplan. Dit gebeurt voor elke PoC. Hierna volgt de uitvoering van deze testplannen op hun respectievelijke PoC. Het is belangrijk dat tijdens de testfase verschillende activiteiten correct worden uitgevoerd. Zo is het noodzakelijk om gedetailleerde notities te hebben in geval van problemen. Een tweede factor die hier nauw bij samenhangt, is de reproduceerbaarheid van de fout of het probleem. Vervolgens zijn de duur en intensiteit van de test belangrijk. Daarom worden de testplannen meerdere malen doorlopen met een vergrotende tijdsduur en belasting- of gebruikintensiteit. Dit alles wordt ook opgenomen in een testverslag dat in de conclusiefase wordt verwerkt. In dit verslag wordt er ook een antwoord geformuleerd op de vraag: \textit{''Hoe kan de proof of concept voor de simulatie worden aangepast om in realiteit te kunnen gebruiken?''}. Dit zal een onderdeel zijn van de conclusie van het testverslag.
\subsection{Realisatiefactor bepaling}
Deze fase is er om een antwoord te bieden op de vraag: \textit{''Vanaf wanneer heeft 5G een commercieel en technologisch voordeel in een ziekenhuis?''}. Deze vraag komt vanuit Citymesh. Als antwoord hierop wordt er gekeken naar de gradatie van het netwerk en de voordelen ten opzichte van de kosten. Deze afweging wordt zowel gemaakt voor kleine als grote ziekenhuizen. Deze afweging wordt dan gebundeld om het kantelpunt te achterhalen. \\ Voor deze bepaling wordt er gebruikgemaakt van zowel installatiekost als onderhoudskosten, maar ook de voordelen die nieuwere systemen hebben ten gevolge van het personeel. 
\subsection{Conclusie}
In de conclusie is er maar één doel. Dit is de onderzoeksvraag beantwoorden: \textit{''Hoe kan een 5G-netwerk als integratie of vervanging van DECT-systemen worden gebruikt voor zorgsimulaties binnen HoGent ?''} In deze fase wordt alles uit vorige fases verzameld en verwerkt tot een concreet antwoord en conclusie op deze vraag. Het eindresultaat zal de bachelorproef thesis zijn. Tenslotte wordt er een korte vergelijking gemaakt van het DECT-systeem met de PoC.
\\\\
Doorheen de hele bachelorproef wordt verwacht dat er een open en directe communicatie is tussen de student en de co-promotoren. Dit is belangrijk omwille van het einddoel van de bachelorproef. Hoewel de onderzoeksvraag een antwoord wil op mogelijke vervanging/integratie, is het de bedoeling dat de tweede PoC in gebruik kan worden genomen door de Hogeschool Gent Departement Gezondheidszorg. Met op het oog om een simulatie te kunnen bieden aan de studenten van een vervangoptie van het DECT-systeem, om alarmmoeheid tegen te gaan.

%---------- Verwachte resultaten ----------------------------------------------
\section{Verwacht resultaat, conclusie}%
\label{sec:verwachte_resultaten}

Het doel van deze thesis is om een oplossing te bieden voor het verouderde DECT-systeem. Dit door eerst een situatieschets voor te brengen van het DECT-systeem. Hiermee worden de voor- en nadelen van het huidige systeem aan het licht gebracht. Als eenmaal de gebreken gekend zijn, kan er gekeken worden om deze op te vangen met een moderner systeem zoals 5G. Er zijn echter verschillende versies van 5G-architecturen. Hoewel er geen officiële standaard is voor 5G-netwerken binnen de gezondheidszorg, zal er een zo objectief mogelijke keuze worden gemaakt met een vergelijking tussen alle mogelijkheden. Een disclaimer is wel noodzakelijk, aangezien het hier gaat om een bachelorproef thesis zal er voornamelijk gekeken worden naar open-source projecten, omwille van budgetten. \\\\
De Proof-of-Concept zal een simulatie zijn van een kleinschalig privaat 5G-netwerk. Hoewel kleinschalig, zal het ontworpen zijn met het oog op schaalbaarheid. Het is te verwachten dat de 5G-oplossing een efficiëntere en accuratere manier voor communicatie zal zijn ten opzichte van het DECT-systeem. Verder wordt er een daling verwacht in het voorkomen van alarmmoeheid. Dit komt omdat er verschillende alarmsignalen gebundeld zullen worden. Er zal ook de mogelijkheid zijn om afbeeldingen of berichten te kunnen sturen, waardoor de arts de situatie van de patiënt beter kan inschatten.\\\\
De volgende stap na deze bachelorproef is de mogelijke integratie van het systeem in de opleidingen binnen het departement Gezondheidszorg van HoGent,om de studenten een alternatief te kunnen aanbieden ten opzichte van het DECT-systeem.
