%---------- Inleiding ---------------------------------------------------------

% TODO: Is dit voorstel gebaseerd op een paper van Research Methods die je
% vorig jaar hebt ingediend? Heb je daarbij eventueel samengewerkt met een
% andere student?
% Zo ja, haal dan de tekst hieronder uit commentaar en pas aan.

%\paragraph{Opmerking}

% Dit voorstel is gebaseerd op het onderzoeksvoorstel dat werd geschreven in het
% kader van het vak Research Methods dat ik (vorig/dit) academiejaar heb
% uitgewerkt (met medesturent VOORNAAM NAAM als mede-auteur).
% 

\section{Inleiding}%
\label{sec:inleiding}

In de zorgindustrie wordt er steeds meer gesproken over 'alarmmoeheid'. Dit is een stijgend probleem. Bij alarmmoeheid wordt het gezondheidszorgpersoneel overspoeld met een constante vloed van alarmsignalen. Deze moeheid is grotendeels te wijten aan het huidige systeem dat in ziekenhuizen wordt gebruikt. Dit systeem is ook wel gekend als het DECT-systeem, een audiosysteem, wat een aantal nadelen met zich meebrengt . De moeheid wordt verder versterkt door het BELET-systeem, het systeem dat de alarmknop bij de patiënt plaatst. De moeheid is echter niet het enige probleem. Zo is er ook de menselijke factor in het systeem. Dit wordt duidelijk met een situatieschets van het gebruik van het DECT-systeem:

\begin{itemize}
  \item patiënt bedient alarmknop
  \item Verplegend personeel wordt verwittigd via het BELET-systeem
  \item Verplegend personeel reageert op alarm en bezoekt de patiënt
  \item Patiënt heeft een situatie die verslechtert en verplegend personeel wil de dokter inschakelen
  \item Verplegend personeel legt mondeling het probleem uit aan de dokter
  \item Dokter maakt oordeel
\end{itemize}

Het is de voorlaatste stap waar een probleem optreedt. De dokter steunt volledig op de beschrijving van het verplegend personeel. Dit betekent dus ook dat eenmaal de conversatie beëindigd is, is er geen naslagwerk om terug op te vallen. Verder is het DECT-systeem verouderd en biedt het geen integratiemogelijkheid van monitoring toestellen.

Deze bachelorproef thesis zal zich focussen op een alternatief voor het DECT-systeem, aan de hand van 5G en is gericht aan IT-personeel in gezondheidszorginstituten. Het primaire doel is om te onderzoeken of het 5G-netwerk een oplossing kan bieden voor de vervanging/modernisering van het DECT-systeem, met als gevolg een daling in alarmmoeheid. Een bijdragende factor hiervoor is het bundelen van alamrsignalen in het 5G-netwerk. Het 5G-netwerk biedt ook extra communicatiemethodes, naast het audiosysteem zoals bij DECT, zoals berichten, foto's , \dots . Op deze manier kan de dokter, stel hij is onbereikbaar omwille van andere interventie, later alles nakijken zonder verder gestoord te worden. Naast het onderzoek zal er een proof of concept worden opgesteld. 

%---------- Stand van zaken ---------------------------------------------------

\section{Literatuurstudie}%
\label{sec:literatuurstudie}

De huidige stand van zaken kan worden opgesplitst in 4 delen. Het eerste deel is een situatieschets van de huidige complete set-up, met al hun voordelen en gebreken. Hierna volgt een uiteenzetting van 5G-frameworks. Hier gaat het niet alleen over algemene frameworks en bemerkingen over 5G, maar ook over meer specifieke gezondheidszorgframeworks van 5G. Nadien wordt er dieper gekeken naar de connectie tussen gezondheidszorg en 5G, en hun interconnecties. Deze interconnecties bevatten onder andere monitoring integratie. Tenslotte is er een korte toelichting op de cybersecurityverplichtingen in kader van deze bachelorproef.
\\
Het DECT-systeem staat voor Digital Enhanced Cordless Telecommunications; dit is een standaard in de EU sinds 1993. De meest gebruikte situatie is waar er meerdere gebruikers zijn voor een draadloze communicatie in werkomgevingen. De voornaamste reden voor gebruik van het DECT-\\draadloos systeem is dat het een grote dichtheid van veel gebruikers aankan. In vergelijking met andere mobile communicatiesystemen werkt het niet buiten de werkomgeving. \autocite{Welinder1997} Na onderzoek van \textcite{Welinder1997} bleek dat de interferentie van het DECT-systeem op medisch gereedschap 11 percent was. De interferentie valt echter volledig weg bij een afstand van 0,5m tussen het gereedschap en het DECT toestel.
\\\\
5G is een stap in de evolutie van het mobiele netwerk, opvolger van 4G. In deze bachelorproefgaat het om het private 5G-netwerk. Volgens \textcite{wen2021private} zijn er eigenschappen van private 5G die nauw aansluiten met die van DECT-systemen. De eerste eigenschap wijst op de hoge device dichtheid, maar bij 5G gaat deze ook nog eens gepaard met hoge throughput. Dit zorgt voor integratie van externe devices, zoals sensoren, camera's, etc. Verdergaand op de gelijkenissen in eigenschappen heeft 5G ook voordelen ten opzichte van DECT-systemen volgens \textcite{wen2021private}. Zo is er een hoge betrouwbaarheid met een lage latency.\\ De tweede eigenschap van het privaat netwerk is de flexibiliteit en voorspelbaarheid van Quality of Service (QoS). Daarnaast zijn er verschillende architecturen die men kan implementeren om een privaat 5G netwerk op te zetten. De eerste methode is een stand-alone deployment. Hierbij wordt een privaat netwerk opgezet, waarbij alle netwerk functies van het netwerk zijn gelimiteerd binnen een logische perimeter bestaande van vooraf gedefinieerde regio's. De andere methode is een public netwerk geintegreerd deployment. Deze architectuur kan worden opgesplitst in verschillende types, afhankelijk van de gradatie van integratie. Deze 4 types zijn de volgende:

\begin{itemize}
  \item O-RAN (Open Radio Access Network)
  \item Gedeelde RAN (Radio Access Network)
  \item Gedeelde RAN en controle vlak
  \item Hosted bij het Publieke Netwerk
\end{itemize}

\textcite{wen2021private} vermeldt ook dat er naast architectuur ook een keuze moet worden gemaakt voor het spectrum. Hier zijn ook opnieuw 3 keuzes: een Dedicated privaat spectrum, een erkend spectrum en een niet-erkend spectrum. 
5G heeft een aantal noodzakelijke technologieën nodig in het netwerk. Een van deze technologieën is network slicing. Hierbij wordt er 'een netwerk in een netwerk' gemaakt door het fysieke netwerk op te splitsen in meerdere logische netwerken. Om aan network slicing te kunnen doen is er een nood aan netwerkvirtualisatie. Het slicing zelf kan worden opgedeeld in 3 lagen: Infrastructuurlaag, Network-slice laag en Onderhoudslaag. De levensloop van het slicen van een netwerk verloopt volgens de volgende 4 fases \autocite{wen2021private}:

\begin{enumerate}
  \item Voorbereiding
  \item In dienst stellen
  \item Gebruik
  \item Ontmanteling
\end{enumerate}

Het derde luik van de literatuurstudie gaat dieper in op de connecties tussen 5G en gezondheidszorg, en hoe deze worden bereikt aan de hand van bestaande frameworks. Er zijn verschillende mogelijkheden, maar met de focus op een toegankelijke methode zal er voornamelijk gefocust worden op open-source frameworks. Zo zijn er verschillende variaties op Open5G. Volgens \textcite{Open5GS2024} is Open5G 'een geavanceerd open-source project ontworpen voor het bouwen en onderhouden van je eigen NR/LTE mobiele netwerk. Open5G Biedt een robuuste oplossing voor het configureren van zowel 5G (NR) als LTE (Evolved) netwerken.' \\ Verder zijn er andere concepten zoals ORAN waar onder andere het OpenCare5G-netwerk is op gebaseerd. In dit framework wordt Open RAN in een privaat netwerk gebruikt voor digitale Gezondheidsapplicaties. Zo hebben \textcite{de2023opencare5g} onderzocht hoe dit framework kan gebruikt worden om gezondheidsonderzoeken te doen met draagbare ultrasone gereedschappen op verschillende locaties. Om deze onderzoeken te kunnen delen met artsen, werd er een lokaal privaat netwerk opgezet. Zo gebruikten ze een 5G xHaul ORAN Operations, Administration en Maintenance (OAM) privaat netwerk. Er wordt gekozen om een top-down systeem te gebruiken. Dit is een georganiseerde manier om een netwerk project te ontwikkelen. Dit komt omdat men kan steunen op het OSI-model en het 5G-RAN protocol layer model. Dit zal ook gebruikt worden bij de methodologie. Zo zal er een analyse zijn van elke laag van de 5G architectuur. Beginnend met de service laag, hier komen de verzoeken van de applicatie van. De architectuur eindigt met de infrastructuurlaag. \autocite{de2023opencare5g}
\\\\
In samenspraak met de co-promotor is er geopteerd om cybersecurity niet actief te onderzoeken in deze bachelorproef thesis. Echter, het is wel noodzakelijk om te weten dat er verschillende verplichtingen zijn in België en de Europese Unie. Hieronder een verduidelijking als context.
\\
Sinds 27 april 2016 is de verordening 2016/679 van toepassing. Deze start de GDPR (General Data Protection Regulation) in de Europese Unie. De verordening 2016/679 vermeldt dat "regels worden vastgelegd betreffende de bescherming van natuurlijke personen in verband met de verwerking van persoonsgegevens en het vrije verkeer van persoonsgegevens. Het beschermt de grondrechten en de fundamentele vrijheden van natuurlijke personen met name hun recht op bescherming van persoonsgegeven."\\ \autocite{gdpr2016} 
\\
Vervolgens is er ook het NIS2, een Belgische wet die bedrijven verplichtingen oplegt op vlak van cybersecurity. "De wet van 26 april 2024 tot vaststelling van een kader voor de cyberbeveiliging van netwerk- en informatiesystemen van algemeen belang voor de openbare veiligheid (de "NIS2-wet") zet de EU-richtlijn 2022/2555 van het Europees Parlement en de Raad van 14 december 2022 (de "NIS2-richtlijn") om." \autocite{Belgium2024}
\\
Tenslotte is er een nieuwer initiatief genaamd EHDS (European Healthcare Data Space). Dit initiatief heeft volgende doel: "De algemene doelstelling is te waarborgen dat natuurlijke personen in de EU in de praktijk meer zeggenschap over hun elektronische gezondheidsgegevens hebben." \autocite{EHDS2022}

% Voor literatuurverwijzingen zijn er twee belangrijke commando's:
% \autocite{KEY} => (Auteur, jaartal) Gebruik dit als de naam van de auteur
%   geen onderdeel is van de zin.
% \textcite{KEY} => Auteur (jaartal)  Gebruik dit als de auteursnaam wel een
%   functie heeft in de zin (bv. ``Uit onderzoek door Doll & Hill (1954) bleek
%   ...'')

%---------- Methodologie ------------------------------------------------------
\section{Methodologie}%
\label{sec:methodologie}

% Alvorens de methodologie volledig wordt besproken; is er een belangrijke disclaimer noodzakelijk te vermelden.
% In samenspraak met de co-promotor is er geopteerd om cybersecurity niet actief te onderzoeken in deze bachelorproef thesis. Echter, het is wel noodzakelijk om te weten dat er verschillende verplichtingen zijn in België en de Europese Unie. In de literatuurstudie is er wel een kort naslagwerk van de verplichtingen ten opzichte van cybersecurity.
% \\\\
Tijdens het verloop van de bachelorproef zullen er antwoorden worden vergaard op verschillende deelvragen die helpen de onderzoeksvraag te beantwoorden. Zo zal elke deelvraag deel uitmaken van een onderdeel van de methodologie.
\\\\
De bachelorproef verloopt in de volgende fasen:

\begin{enumerate}
  \item Research / Voorbereiding
  \item In dienst stellen / Setup
  \item Gebruik en Testen
  \item Conclusie
\end{enumerate}

\textbf{1. Research / Voorbereiding Fase}\\
De eerste vraag die gesteld moet worden is \textit{''Wat moet het 5G netwerk vervangen ?''} \\
Hiermee wordt de onderzoeksfase gestart. De eerste stap van de onderzoeksfase is het in kaart brengen van het huidige DECT-systeem. Dit zal bekeken worden vanuit een algemeen overzicht. Het doel is om een topologisch schema en een duidelijk overzicht te hebben van de werking van een DECT-systeem.\\
Vervolgens wordt er een opvolgvraag gesteld, \textit{''Hoe vervangen we het DECT-systeem?''}\\ Deze vraag kan enkel beantwoord worden na een kort onderzoek in de verschillende frameworks, architecturen. Als besluit van dit onderzoek zal er een vergelijkende studie zijn. Hieruit wordt het best passende framework gekozen.
Als vervolg op de keuze van het framework, zullen de technische specificaties worden vastgelegd en opgelijst. Dit is een noodzakelijke stap om in de conclusie op het einde een duidelijk beeld te kunnen maken van mogelijke up-scaling.\\
De laatste deelvraag van de research / voorbereidingsfase is \textit{''Hoe moet het 5G-netwerk eruitzien, rekening houdend met framework en scaling?''}\\ Hoewel dit de laatste stap is in de voorbereidende fase, is deze cruciaal voor alle verdere fases. Hier wordt alle informatie vergaard in de Research fase en in een netwerkschema / topologie verwerkt. Dit zal als leidraad worden gebruikt in het verdere verloop van de bachelorproef.\\\\
\textbf{2. In dienst stellen / Setup Fase}\\
Deze fase bestaat uit het omzetten van de verworven theorie naar een Proof-of-Concept (PoC). Deze bachelorproef zal bestaan uit 2 PoC's. De eerste zal een simulatie zijn die lokaal op de pc werkt. Het tweede deel is een praktische realisatie, met de artikelen opgelijst in de technische specificaties. \\\\
\textbf{3. Gebruik en Testfase}\\
Het doel van deze fase is het antwoorden op de vraag \textit{''Wat kan er fout gaan en hoe kan dit worden vermeden?''} De gebruik- en testfase verloopt net zoals de setup fase in twee delen. Het eerste deel is het opstellen van een eenduidig en gedetailleerd testplan. Dit gebeurt voor elke PoC. Hierna volgt de uitvoering van deze testplannen op hun respectievelijke PoC. Het is belangrijk dat tijdens de testfase verschillende activiteiten correct worden uitgevoerd. Zo is het noodzakelijk om gedetailleerde notities te hebben bij geval van problemen. Een tweede factor die hier nauw bij samenhangt, is de reproduceerbaarheid van de fout of het probleem. Vervolgens is de duur en intensiteit van de test belangrijk. Daarom worden de testplannen meerdere malen doorlopen met een vergrotende tijdsduur en belasting- of gebruikintensiteit. Dit alles wordt ook opgenomen in een testverslag dat in de conclusiefase wordt verwerkt.\\\\
\textbf{4. Conclusie}\\
In de conclusie is er maar één doel. Dit is de onderzoeksvraag beantwoorden: \textit{''Hoe kan een 5G netwerk als integratie of vervanging van DECT-systemen worden gebruikt voor zorgsimulaties binnen HoGent ?''} In deze fase wordt alles uit vorige fases verzameld en verwerkt tot een concreet antwoord en conclusie op deze vraag. Het eindresultaat zal de bachelorproef thesis zijn. Tenslotte wordt er een korte vergelijking gemaakt van het DECT-systeem met de PoC.
\\\\
Doorheen de hele bachelorproef wordt verwacht dat er een open en directe communicatie is tussen de student en de co-promotor. Dit is belangrijk omwille van het einddoel van de bachelorproef. Hoewel de onderzoeksvraag een antwoord wil op mogelijke vervanging/integratie, is het de bedoeling dat de tweede PoC in gebruik kan worden genomen door de Hogeschool Gent Departement Gezondheidszorg. Met op het oog om een simulatie te kunnen bieden aan de studenten van een vervangoptie van het DECT-systeem, om alarmmoeheid tegen te gaan.

%---------- Verwachte resultaten ----------------------------------------------
\section{Verwacht resultaat, conclusie}%
\label{sec:verwachte_resultaten}

Het doel van deze thesis is om een oplossing te bieden voor het verouderde DECT-systeem. Dit door eerst een situatieschets voor te brengen van het DECT-systeem. Hiermee worden de voor- en nadelen van het huidige systeem aan het licht gebracht. Als eenmaal de gebreken gekend zijn, kan er gekeken worden om deze op te vangen met een moderner systeem zoals 5G. Er zijn echter verschillende versies van 5G-architecturen. Hoewel er geen officiële standaard is voor 5G-netwerken binnen gezondheidszorg, zal er een zo objectief mogelijke keuze worden gemaakt met een vergelijking tussen alle mogelijkheden. Een disclaimer is wel noodzakelijk, aangezien het hier gaat om een bachelorproef thesis zal er voornamelijk gekeken worden naar open-source projecten, omwille van budgetten. \\\\
De Proof-of-Concept zal een simulatie zijn van een kleinschalig privaat 5G-netwerk. Hoewel kleinschalig, zal het onworpen zijn met het oog op schaalbaarheid. Het is te verwachten dat de 5G oplossing een efficiëntere en accuratere manier voor communicatie zal zijn ten opzichte van het DECT-systeem. Verder wordt er een daling verwacht in het voorkomen van alarmmoeheid. Dit komt omdat er verschillende alarmsignalen gebundeld zullen worden. Er zal ook de mogelijkheid zijn om afbeeldingen of berichten te kunnen sturen, waardoor de arts de situatie van de patiënt beter kan inschatten.\\\\
De volgende stap na deze bachelorproef is de mogelijke integratie van het systeem in de opleidingen binnen het departement Gezondheidszorg van HoGent,om de studenten een alternatief te kunnen aanbieden ten opzichte van het DECT-systeem.