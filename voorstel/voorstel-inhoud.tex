%---------- Inleiding ---------------------------------------------------------

% TODO: Is dit voorstel gebaseerd op een paper van Research Methods die je
% vorig jaar hebt ingediend? Heb je daarbij eventueel samengewerkt met een
% andere student?
% Zo ja, haal dan de tekst hieronder uit commentaar en pas aan.

%\paragraph{Opmerking}

% Dit voorstel is gebaseerd op het onderzoeksvoorstel dat werd geschreven in het
% kader van het vak Research Methods dat ik (vorig/dit) academiejaar heb
% uitgewerkt (met medesturent VOORNAAM NAAM als mede-auteur).
% 

\section{Inleiding}%
\label{sec:inleiding}

In de zorgindustrie wordt er steeds meer gesproken over 'Alarm moeheid.' Dit is een stijgend probleem. Bij alarm moeheid wordt het gezondheidszorgpersoneel overspoeld met een constante vloed van alarmsignalen. Deze moeheid is grotendeels te wijten aan het huidige systeem dat in ziekenhuizen wordt gebruikt. Dit systeem is ook wel gekend als het DECT-systeem. Dit systeem is een audio-systeem, hier zijn de meeste nadelen aan verbonden. Zoals eerder vermeld zullen de overvloed aan alarmsignalen, onder andere gegenereerd door het DECT-systeem, alarm moeheid veroorzaken. Deze moeheid wordt ook versterkt door het belet-systeem, het systeem dat de alarmknop bij de patient plaatst. De moeheid is echter niet het enige probleem. Verder is er een groter probleem en dat is de menselijke factor aan het systeem. Dit wordt duidelijk met een situatie-schets van het gebruik van het DECT-systeem:

\begin{itemize}
  \item Patient bedient alarm knop
  \item Verplegend personeel wordt verwittigd via BELET-systeem
  \item Verplegend personeel reageert op alarm en bezoekt patient
  \item Patient heeft een situatie die verslechtert en verplegend personeel wil dokter inschakelen
  \item Verplegend personeel legt mondeling het probleem uit aan de dokter
  \item Dokter maakt oordeel
\end{itemize}

Het is de voorlaatste stap waar een probleem optreedt. De dokter steunt volledig op de beschrijving van het verplegend personeel. Dit betekent dus ook dat eenmaal de conversatie beeindigd is, is er geen naslagwerk om terug op te vallen.

Deze bachelorproef thesis zal zich focussen op een alternatief voor het DECT-systeem, aan de hand van 5G en is gericht aan IT-personeel in gezondheidszorginstituten. Het primaire doel is om te onderzoeken of het 5G-netwerk een oplossing kan bieden voor de vervanging/modernisering van het DECT-Systeem. Naast het onderzoek zal er een proof of Concept worden opgesteld.

%---------- Stand van zaken ---------------------------------------------------

\section{Literatuurstudie}%
\label{sec:literatuurstudie}

De huidige stand van zaken kan worden opgesplitst in 4 delen. Het eerste deel een situatie schets van de huidige complete set-up, met al hun voordelen en gebreken. Hierna volgt een uiteenzetting van 5G-frameworks, hier gaat het niet alleen over algemene frameworks en bemerkingen over 5G maar ook al meer specifiek gezondheidszorg frameworks van 5G. Nadien wordt er dieper gekeken naar de connectie tussen gezondhseidszorg en 5G, maar ook hun interconnecties. Deze interconnecties bevatten onder andere monitoring integratie. Tenslotte is er een korte toelichting van de cybersecurity verplichtingen in kader van deze bachelorproef.
\\\\
Het DECT-systeem staat voor Digital Enhanced Cordless Telecommunications, dit is een standaard in EU sinds 1993. De meest gebruikte situatie is waar er meerdere gebruikers zijn voor een draadloze communicatie in werkomgevingen. De voornaamste reden voor gebruik van het DECT draadloos systeem is dat het een grote dichtheid van veel gebruikers aankan. Terwijl het in vergelijking met andere mobile communicatie systemen niet werkt buiten de werkomgeving.\autocite{Welinder1997} Na onderzoek van \textcite{Welinder1997} bleek dat de interferentie van het DECT-systeem op medisch gereedschap  was 11 percent. Maar de interferentie valt volledig weg bij een afstand van 0.5m tussen het gereedschap en het DECT toestel.
\\\\
5G is een stap in de evolutie van het mobiele netwerk, opvolger van 4G. Nu het 5G netwerk waar het om draait in deze bachelorproef is het private 5G netwerk. Volgens \textcite{wen2021private} zijn er eigenschappen van private 5G die nauw aansluiten met die van DECT-systemen. De eerste is de hoge device dichtheid, maar bij 5G gaat deze ook nog eens gepaart met hoge throughput. Dit zorgt ervoor dat integratie van externe devices soals sensoren, camera's, etc. Verdergaant op de gelijkenissen in eigenschappen heeft 5G ook voordelen ten opzichte van DECT-systemen volgens \textcite{wen2021private}. Zo Is er een hoge betrouwbaarheid met een lage latency.\\ Verder heeft een privaat netwerk nog het extra voordeel van een aanpasbare voorspelbare Quality of Service (QoS). Daarnaast zijn er verschillende architecturen die men kan implementeren om een privaat 5G netwerk op te zetten. De eerste is een stand-alone deployment, hierbij wordt privaat netwerk opgezet waarbij alle netwerk functies van het netwerk zijn gelimiteerd binnen een logische perimeter bestaande van vooraf gedefinieerde regios. De andere manier is een public netwerk geintegreerd deployment. Deze architectuur kan worden opgesplitst in 3 types, afhankelijk van de gradatie van integratie. Deze 4 types zijn de volgende:

\begin{itemize}
  \item O-RAN (Open Radio Access Network)
  \item Gedeelde RAN (Radio Access Network)
  \item Gedeelde RAN en controle vlak
  \item Hosted bij het Publieke Netwerk
\end{itemize}

\textcite{wen2021private} vermeldt ook dat er naast architectuur ook een keuze moet gemaakt worden voor het spectrum. Hier zijn ook opnieuw 3 keuzes: een Dedicated pirvaat spectrum, een erkend spectrum en een niet-erkend spectrum. 
5G heeft een aantal noodzakelijke technologiën nodig in het netwerk. Een van deze technologiën is network slicing. Hierbij wordt er 'een netwerk in een netwerk' gemaakt door het fysieke netwerk op te splitsen in meerdere logische netwerken. Om aan network slicing te kunnen doen is er een nood aan netwerk virtualisatie. Het silcing zelf kan worden opgedeeld in 3 lagen: Infrastructuur laag, Network slice laag en Onderhouds laag. De levensloop van het slicen van een netwerk verloopt volgens volgende 4 fases \autocite{wen2021private}:

\begin{enumerate}
  \item Voorbereiding
  \item In dienst stellen
  \item Gebruik
  \item Ontmanteling
\end{enumerate}

Het derde luik van de literatuurstudie gaat dieper in op de connecties tussen 5G en gezondheidszorg, en hoe deze worden bereikt aan de hand van bestaande frameworks.
\\\\
In samenspraak met de co-promotor is er geopteerd om cybersecurity niet actief te onderzoeken in deze bachelorproef thesis. Echter het is wel noodzakelijk om te weten dat er verschillende verplichtingen zijn in België en de Europese Unie. Hieronder een kleine verduidelijking als context.
\\
Sinds 27 april 2016 is de verordening 2016/679 van toepassing. Deze start de GDPR (General Data Protection Regulation) in de Europese unie. De verordening 2016/679 vermeldt dat "regels worden vastgelegd betreffende de bescherming van natuurlijke personen in verband met de verwerking van persoonsgegevens en het vrije verkeer van persoonsgegevens. En het beschermt het ook de grondrechten en de fundamentele vrijheden van natuurlijke personen met name hun recht op bescherming van persoonsgegeven."\\ \autocite{gdpr2016} 
\\
Vervolgens is er ook het NIS2, dit is een Belgische wet die bedrijven verplichtingen oplegt op vlak van cybersecurity. "De wet van 26 april 2024 tot vaststelling van een kader voor de cyberbeveiliging van netwerk- en informatiesystemen van algemeen belang voor de openbare veiligheid (de "NIS2-wet") zet de EU-richtlijn 2022/2555 van het Europees Parlement en de Raad van 14 december 2022 (de "NIS2-richtlijn") om." \autocite{Belgium2024}
\\
Tenslotte is er een nieuwer initiatief genaamd EHDS (European Healthcare Data Space). Dit initiatief heeft als algemene doelstelling: "De algemene doelstelling is te waarborgen dat natuurlijke personen in de EU in de praktijk meer zeggenschap over hun elektronische gezondheidsgegevens hebben." \autocite{EHDS2022}

% Voor literatuurverwijzingen zijn er twee belangrijke commando's:
% \autocite{KEY} => (Auteur, jaartal) Gebruik dit als de naam van de auteur
%   geen onderdeel is van de zin.
% \textcite{KEY} => Auteur (jaartal)  Gebruik dit als de auteursnaam wel een
%   functie heeft in de zin (bv. ``Uit onderzoek door Doll & Hill (1954) bleek
%   ...'')

%---------- Methodologie ------------------------------------------------------
\section{Methodologie}%
\label{sec:methodologie}

Alvorens de methodologie volledig wordt besproken is er een belangrijke disclaimer noodzakelijk te vermelden.
In samenspraak met de co-promotoris er geopteerd om cybersecurity niet actief te onderzoeken in deze bachelorproef thesis. Echter het is wel noodzakelijkom te weten dat er verschillende verplichtingen zijn in Belgie en de Europese Unie. In de literatuur studie is er wel een kort naslagwerk van de verplichtingen ten opzichte van cybersecurity.
\\\\
Hier beschrijf je hoe je van plan bent het onderzoek te voeren. Welke onderzoekstechniek ga je toepassen om elk van je onderzoeksvragen te beantwoorden? Gebruik je hiervoor literatuurstudie, interviews met belanghebbenden (bv.~voor requirements-analyse), experimenten, simulaties, vergelijkende studie, risico-analyse, PoC, \ldots?

Valt je onderwerp onder één van de typische soorten bachelorproeven die besproken zijn in de lessen Research Methods (bv.\ vergelijkende studie of risico-analyse)? Zorg er dan ook voor dat we duidelijk de verschillende stappen terug vinden die we verwachten in dit soort onderzoek!

Vermijd onderzoekstechnieken die geen objectieve, meetbare resultaten kunnen opleveren. Enquêtes, bijvoorbeeld, zijn voor een bachelorproef informatica meestal \textbf{niet geschikt}. De antwoorden zijn eerder meningen dan feiten en in de praktijk blijkt het ook bijzonder moeilijk om voldoende respondenten te vinden. Studenten die een enquête willen voeren, hebben meestal ook geen goede definitie van de populatie, waardoor ook niet kan aangetoond worden dat eventuele resultaten representatief zijn.

Uit dit onderdeel moet duidelijk naar voor komen dat je bachelorproef ook technisch voldoen\-de diepgang zal bevatten. Het zou niet kloppen als een bachelorproef informatica ook door bv.\ een student marketing zou kunnen uitgevoerd worden.

Je beschrijft ook al welke tools (hardware, software, diensten, \ldots) je denkt hiervoor te gebruiken of te ontwikkelen.

Probeer ook een tijdschatting te maken. Hoe lang zal je met elke fase van je onderzoek bezig zijn en wat zijn de concrete \emph{deliverables} in elke fase?

%---------- Verwachte resultaten ----------------------------------------------
\section{Verwacht resultaat, conclusie}%
\label{sec:verwachte_resultaten}

Hier beschrijf je welke resultaten je verwacht. Als je metingen en simulaties uitvoert, kan je hier al mock-ups maken van de grafieken samen met de verwachte conclusies. Benoem zeker al je assen en de onderdelen van de grafiek die je gaat gebruiken. Dit zorgt ervoor dat je concreet weet welk soort data je moet verzamelen en hoe je die moet meten.

Wat heeft de doelgroep van je onderzoek aan het resultaat? Op welke manier zorgt jouw bachelorproef voor een meerwaarde?

Hier beschrijf je wat je verwacht uit je onderzoek, met de motivatie waarom. Het is \textbf{niet} erg indien uit je onderzoek andere resultaten en conclusies vloeien dan dat je hier beschrijft: het is dan juist interessant om te onderzoeken waarom jouw hypothesen niet overeenkomen met de resultaten.

